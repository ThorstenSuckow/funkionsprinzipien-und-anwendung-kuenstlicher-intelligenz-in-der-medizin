\chapter{Zusammenfassung und Bemerkung}



In Kapitel~\ref{ch:neuron} haben wir gesehen, dass ein biologisches neuronales Netz ein Kommunikationsverbund ist, in dem hochkomplexe biochemische Prozesse ineinandergreifen.
Diese Prozesse ermöglichen eine Informationsverarbeitung: Eine einzelne Nervenzelle ist in der Lage, Signale zu empfangen, zu verrechnen und ein Signal auszugeben.
Löst ein Aktionspotenzial aus, diffundieren Neurotransmitter in den synaptischen Spalt, die Folgezelle verrechnet die daraus entstehenden elektrischen Signale und so setzt sich die Kette fort, bis tausende von Neuronen ein Signal an das Ende der Kommunikationskette weitergeleitet haben.
Ob ein einzelnes Vesikel ein \textbf{mEPSC} verursacht, mag dabei auf analytischer Ebene interessant sein.
Die Frage, welches der oft über tausend Vesikel, die in einem Neuronenverband an einer Signalweiterleitung beteiligt sind, ursächlich für eine muskuläre Reaktion ist, ist sicherlich schwieriger zu beantworten.

Tiefe künstliche neuronale Netz mit ihren Millionen von Parametern (vgl.~\cite{SVI+15}) erweisen sich im Hinblick auf eine Analyse ähnlich undurchschaubar\footnote{ dieser Blackbox-Charakter wird problematisch, wenn {bspw.} von Assistenzsystemen in der Medizin Entscheidungswege aufgezeigt werden sollen, die zu einer bestimmten Diagnose oder Therapieempfehlung führen (siehe Abschnitt~\ref{sec:diagnostik}).}.\\

Die Modellierung solcher Netze folgt dabei in Teilen dem biologischen Vorbild, und es mag zunächst überraschen, dass einige solcher Verschaltungsmuster - wie die \textbf{Divergenz} oder die \textbf{rekurrente} oder \textbf{laterale Hemmung} - zunächst an Logikgatter erinnern (vgl.~\cite[58 f.]{Eil19}). Es liegt nahe, dass die Wissenschaft versucht ist, solche naturgegebenen verrechnenden Einheiten und logischen Verschaltungen zu verstehen und in mathematische Formeln zu gießen, um ein Modell zu erstellen, das menschliche Intelligenz erklärt und nachstellt, und dann als Abstraktionen auf Maschinen übertragen wird: ``if you really understand something, you can usually make a machine do it``~\cite[xiii]{AR88}.\\

Einer dieser Versuche wurde in Abschnitt~\ref{sec:mcpneuron} mit dem MCP-Modell vorgestellt, ein \textbf{empirisches Modell}\footnote{
    auch ``caricature model``; beides ~\cite[4]{HI97}
}, das auf Analyse und einfacher Schwellenwertlogik basiert (vgl.~\cite[16]{AR88}).

In dem dort zitierten Papier weisen McCulloch und Pitts darauf hin, dass das Alles-Oder-Nichts-Prinzip ({vgl.} Abschnitt~\ref{sec:synaptischeuebertragung}) als Vorbild für ihre Abstraktion durch Aussagenlogik dient.
Das von ihnen erstellte Modell ist dabei eine starke Vereinfachung des biologischen Vorbilds und weist dadurch eine äußerst nützliche Eigenschaft auf:

\blockquote[{\cite[43; Hervorhebung i. O.]{BJ90}}]{
    This ensures its status as a \textit{model}, and not a \textit{copy}, of a real neuron, and makes it possible to implement on a digital computer.
}.

Dabei wurde ein anderer neurobiologischer Prozess, der heutzutage bei maschinellen Lernverfahren Standard ist, nicht von ihnen berücksichtigt: Die dynamische Anpassung des Netzes \footnote{
    ``Lernen`` geschieht auf physiologischer Ebene durch die Modulierung von Synapsen~\cite[115]{HS19c}; McCulloch und Pitts setzen in ihrem Kalkül voraus, dass sich das Netz nicht verändert (siehe Abschnitt~\ref{sec:mcpkalkül}).
}.

McCulloch und Pitts sind sich all dessen durchaus bewusst\footnote{
    ``McCulloch and Pitts acknowledged in their paper that their definition of a neuron was idealized, and that they made physical assumptions that were 'most convenient for the calculus'``~\cite[21]{Abr02}. Siehe auch~\cite[101]{MP43}: ``we regard facilitation and extinction as dependent upon continuous changes in threshold related to electrical and chemical variables``, sowie ``He {[McCulloch]} never claimed that the 1943 model exhausted the richness of individual neurons``~\cite[11]{Arb19}
}.
Entsprechend ist ihr Modell statisch und muss zur Bewältigung von Aufgaben zunächst durch Änderung der Netztopologie {bzw.} Anpassung einzelner Schwellenwerte ``konstruiert`` werden (vgl.~\cite[28]{Fau94} sowie~\cite[51]{Roj93}), selbständig zu lernen vermag es nicht.\\


Die McCulloch-Pitts-Zelle ist von eher geringer Bedeutung für die Neurowissenschaft gewesen\footnote{
    ``The immense theoretical influence of this paper was not among neuroscientists but among computer scientists.``~\cite[17]{AR88}; \textit{Rojas} kommt zu einem ähnlichen Schluss: ``Die Art der Schaltungen, die mit diesen Zellen gebaut werden, ist aber biologisch gesehen nicht so relevant.``~\cite[51]{Roj93}
}.
Umso größer war ihr Einfluss auf die Computerwissenschaften\footnote{
    Die Arbeit war ebenso wichtig für die Entwicklung des ``Konnektionismus``(vgl.~\cite[11]{Arb19}), einer Forschungsrichtung der künstlichen Intelligenz, in der Modelle (neuronale Netze) untersucht werden, mit deren Hilfe sich intelligente und kognitive Handlungen auf Maschinen übertragen lassen (vgl.~\cite[v]{Dor91})
}: Logische Schaltungen, die Intelligenz und Kognition erklären sollen, brachten den Gedanken an eine ``intelligente Maschine`` hervor (vgl.~\cite[204]{Pic04}): So war diese Formalisierung ein wichtiges Schlüsselelement für die von-Neumann-Rechnerarchitektur\footnote{siehe \cite{Neu93}} sowie Wieners Kybernetik\footnote{siehe Anhang~\ref{appendix:wiener}}, und zusammen mit den nachfolgenden Arbeiten von \textit{Hebb}\footnote{siehe ~\cite{Heb49}} und \textit{Rosenblatt} ebnete es den Weg für die Forschung an künstlicher Intelligenz (vgl.~\cite[1]{Arb19}).


Erwähnenswert ist auch die Bemerkung von McCulloch und Pitts bzgl. ``Berechenbarkeit``:

\blockquote[{~\cite[113]{MP43}}]{
    This is of interest as affording a psychological justification of the Turing definition of computability and its equivalents, Church’s A-definability and Kleene’s primitive recursiveness: if any number can be computed by an organism, it is computable by these definitions, and conversely.
}


Sie nehmen hiermit Bezug auf die von \textit{Alan Turing} (1912 - 1954) bereits 1936 veröffentlichte Arbeit~\cite{Tur37}, in der Turing die Beschreibung der Operationen seines Computers einleitet mit:

\blockquote[{~\cite[250]{Tur37}}]{
    Let us imagine the operations performed by the computer to be split up into 'simple operations' which are so elementary that it is not easy to imagine them further divided.
}


Parallelen zu der Architektur der von Turing aufgestellten Modelle und dem von McCulloch entworfenen ``psychon`` als kleinste Einheit psychischer Aktivitäten sind erkennbar. \textit{Piccinini} stellt dazu fest, dass die Aussage, MCP-Netze können Berechnungen anstellen, einen ersten wichtigen Bezug zwischen mathematischer Theorie und Neurowissenschaft herstellte (vgl.~\cite[197]{Pic04}).\\


McCulloch schreibt dazu später

\blockquote[{\cite[164]{Mcc16}}]{
    Pitts and I showed that brains were Turing machines, and that any Turing machine could be made out of neurons.
}

\noindent
und stellt fest, dass das Nervensystem eine ausgezeichnete logische Maschine repräsentiert (vgl.~\cite[80]{Mcc16}).\\


Konsequenterweise nutzt \textit{Minsky} in~\cite[32 ff.]{Min67} die MCP-Zelle als \textit{endlichen Automaten}\footnote{
    Ein endlicher Automat ist eine informationsverarbeitende Maschine. Basierend auf Eingaben lassen sich mit einer solchen Maschine Zustände, Zustandsübergänge und Ausgaben modellieren. Bei einem \textit{endlichen} Automaten sind die genannten Eingaben, Zustände, und Ausgaben endlich (vgl.~\cite[26 ff.]{SSH95}).
}\footnote{
    dass MCP-Netze endliche Automaten sind: Auch in~\cite[76]{Cow90} sowie~\cite[47, ``Satz 2.4``]{Roj93}. \textit{Arbib} weist darauf hin, dass die Arbeit von McCulloch und Pitts grundlegend für Automatentheorie gewesen ist (vgl.~\cite[8]{Arb19}).
} mit zwei Zuständen, aus denen andere endliche Automaten gebaut werden, und nennt diese, im Jahr 1967 schon geläufig: \textbf{Neuronale Netze}~\cite[33]{Min67}.

%

\textit{Minsky} analysiert ebenfalls das im Abschnitt~\ref{sec:rosenblattperceptron} vorgestellte Perzeptron.


In~\cite{MP88} untersucht er zusammen mit \textit{Papert} u.a. das Verhalten des Perzeptrons im Fall nicht-separierbarer Daten\footnote{
    in~\cite[181 ff.]{MP88}
} sowie das Problem bzgl. \textit{recognition of connectedness}\footnote{
    s. \cite[12, ``Theorem 0.8``]{MP88} sowie~\cite[249 f.]{MP88}. In dem Epilog, der 19 Jahre nach der Erstveröfffentlichung geschrieben wird, machen sie deutlich, dass sie nicht nachweisen wollten, dass ein Perzeptron nicht in der Lage zu der Erkennung zusammenhängender geometrischer Figuren sei, sondern dass die Komplexität von verschiedenen Aufgaben Anforderungen an ein Perzeptron stellt, die zu der damaligen Zeit (Ende 1950er / Anfang 1960er) schwierig umzusetzen gewesen sind (vgl. ~\cite[250]{MP88})
}, bei dem es um die Erkennung zusammenhängender geometrischer Figuren geht.

Ihr Anliegen mit dem Buch ist es, die (mathematischen) Grenzen des Rosenblatt-Perzeptrons auf ein formales Gerüst zu stellen (vgl.~\cite[249]{MP88}).
Allerdings verstärken ihre Ausführungen die zu diesem Zeitpunkt ohnehin schon skeptische Haltung\footnote{
    Here was a machine that could do pattern recognition in a humanlike way; it could recognize all kinds of things. Almost everyone at MIT was very skeptical``~\cite[99]{AR98}
} gegenüber der Fähigkeiten des Perzeptrons\footnote{
    auch wegen schlechter Skalierbarkeit des Modells in der Praxis (vgl.~\cite[159]{AR88})
} und künstlicher neuronaler Netze allgemein, was \textit{Anderson und Rosenfeld} rückblickend auch auf Aussagen in der Einleitung des Buches zurückführen wie\footnote{
    Auch eine generelle Irritation und Enttäuschung über das Perzeptron-Modell in Hinblick auf mehrschichtige Modelle meinen \textit{Anderson und Rosenfeld} in ``we [Minsky und Papert] consider it to be an important research problem to elucidate (or reject) our intuitive judgement that the extension is sterile.``~\cite[232]{MP88} zu erkennen (vgl.~\cite[159]{AR88}).
}

\blockquote[{~\cite[19]{MP88}}]{
    hundreds of projects and experiments [bzgl. des Perzeptrons] were generally disappointing, and the explanations inconclusive. The machines usually work quite well on very simple problems but detoriate very rapidly as the tasks assigned to the get harder.
}

\noindent
Diesen ``Pessimismus`` betrachten \textit{Minsky und Papert} im Jahr 1988 in Retrospektive (vgl.~\cite[xiii]{MP88}). Sie sind sich der Behauptung bewusst, ihr Buch hätte mit dem Aufzeigen der Grenzen des Rosenblatt-Modells den Forscherdrang an maschinellem Lernen gebremst:

\blockquote[{~\cite[xii]{MP88}}]{
    One popular version is  that the publication of our book so discouraged research on learning in network machines that a promising line of research was interrupted.
}

\noindent
\textit{Anderson und Rosenfeld} erwähnen ``the influence of Marvin Minsky and Seymour Papert on the loss of interest in neural networks during the 1970s``~\cite[X]{AR98}.

\textit{Russell und Norvig} fassen zusammen, dass \textit{Minsky und Papert} in ihrem Buch bewiesen haben, dass ein Perzeptron alles lernen kann, was es auch darstellen kann, aber es könnte halt nur sehr wenig darstellen (vgl.~\cite[45]{RN09})\footnote{
    vgl. hierzu~\cite[xiii]{MP88}
}.
Auf gleicher Seite verweisen sie auf die zunehmende Komplexität der zu berechnenden Modelle, die nicht alleine durch schnellere und bessere Hardware kompensiert werden konnte.

Der Lighthill Report\footnote{
    ``Workers entered the field around 1950, and even around 1960, with high hopes that are very far from having been realised in 1972. In no part of the field have the discoveries made so far produced the major impact that was then promised.`` in \url{https://www.chilton-computing.org.uk/inf/literature/reports/lighthill\_report/p001.htm}, ``3 Past disappointments``, abgerufen 28.08.2023
} wird 1973 die britischen Regierung dazu bewegen, das Budget für die Forschung an KI zu kürzen (vgl. ~\cite[45]{RN09}): Neuronale Netze werden als Grundlage künstlicher Intelligenz zunächst verworfen\footnote{
    vgl.~\cite[641]{Ola96}
}, und Forschungsarbeiten an ihnen geht bis zum Anfang der 1980er Jahre zurück.
Diese Periode ist gemeinhin als ``KI-Winter`` (siehe Anhang~\ref{appendix:kiwinter}) bekannt.\\

Die in Kapitel~\ref{ch:knn} vorgestellten Architekturen und Algorithmen sorgten dann zusammen mit dem technologischen Fortschritt für ein erneutes Aufleben der Forschung an neuronalen Netzen und Künstlicher Intelligenz in den 1980er Jahren.
Einige der Ergebnisse dieser Anstrengung wurden in Kapitel~\ref{ch:gesundheitswesen} in Auszügen vorgestellt.
Dass dabei die Algorithmen nicht immer der Funktionsweise des natürlichen Vorbildes entsprechen, haben wir bei dem Backpropagation-Verfahren gesehen (siehe Abschnitt~\ref{sec:backpropagation}).

Bei allem Fortschritt, der durch Forschung und Wissenschaft in den vergangenen Jahren erreicht wurde, ist bei den hier vorgestellten Ergebnissen aber vor allem eins deutlich: Die hohe Leistungsfähigkeit einiger dieser Netze kann nur durch Lernen erreicht werden (vgl.~\cite[40]{AHR19}), wozu umfangreiche, qualitativ hochwertige Daten benötigt werden.

\textit{Nguyen und Patrick} stellen in~\cite{NP16} fest, dass klinische Daten aufgrund verschiedener Faktoren wie Heterogenität (maschinell / handschriftlich erstellte Daten) und weiterem Rauschen, wie unbekannten Abkürzungen oder Rechtschreibfehlern, als Trainingsdaten für maschinelles Lernen eine besondere Herausforderung darstellen.

Normierte Daten können deshalb helfen, die Auswertung, Anwendung aber auch den Austausch derselben zu erleichtern, wie das Beispiel aus Abschnitt~\ref{sec:therapieprognose} zeigt (vgl. auch ~\cite[42]{AHR19}).

Wir dürfen schließen, dass Künstliche Intelligenz in Form neuronaler Netze nicht nur durch komplexe mathematische Modelle und hohe Rechenleistung ermöglicht wird: Sie benötigt zu der erfolgreichen Bewältigung ihrer Aufgaben auch eine gesunde Datenbasis für ihre Lernverfahren\footnote{
    \textit{Dash et al} fassen in~\cite{SSM+19} einige wesentliche Punkte zum Thema Big Data im Gesundheitswesen zusammen, ausserdem \textit{Rüping und Sander} in~\cite{RS19}, die dort feststellen: ``Bei der datengetriebenen Entwicklung von Systemen ist zu beachten, dass die Qualität des Ergebnisses zentral von der Qualität der Eingabedaten abhängt.``~\cite[19]{RS19}
}.













%
%
%
%

