\chapter{Zusammenfassung und Bemerkung}



\subsection{Zusammenfassung MCP}\label{mcp-summary}

Das MCP-Modell ist ein \textbf{empirisches Modell}\footnote{
    auch ``caricature model``; beides ~\cite[4]{HI97}
} und basiert auf Analyse und einfacher Schwellenwertlogik~\cite[16]{AR88}.
In ihrem Papier weisen McCulloch und Pitts darauf hin, dass das Alles-Oder-Nichts-Prinzip als Vorbild für ihre Abstraktion durch Aussagenlogik dient.
Von allem, was wir in Kapitel 1 über Nervenzellen erfahren haben, dürfen wir den Schluss ziehen, dass das von ihnen erstellte Modell eine starke Vereinfachung ist\footnote{
    ``This ensures its status as a \textit{model}, and not a \textit{copy}, of a real neuron, and makes it possible to implement on a digital computer.``~\cite[43; Hervorhebung i. O.]{BJ90}
}.
Aktionspotenzialen lassen sich gewiss nicht auf zweiwertige Logik reduzieren, darüber hinaus berücksichtigen sie auch nicht eine mögliche Veränderung des Netzes, z.B. durch Lernen\footnote{
    ``Lernen`` geschieht auf physiologischer Ebene durch die Modulierung von Synapsen~\cite[115]{HS19c}
}.
McCulloch und Pitts sind sich dessen durchaus bewusst\footnote{
    ``McCulloch and Pitts acknowledged in their paper that their definition of a neuron was idealized, and that they made physical assumptions that were 'most convenient for the calculus'``~\cite[21]{Abr02}. Siehe auch~\cite[101]{MP43}: ``[...]we regard facilitation and extinction as dependent upon continuous changes in threshold related to electrical and chemical variables, [...]``, sowie ``He {[McCulloch]} never claimed that the 1943 model exhausted the richness of individual neurons``~\cite[11]{Arb19}
}.
Es bleibt ein statisches Modell, es muss ``konstruiert`` werden (vgl. ~\cite[28]{Fau94}).
Es ist nur durch Änderung der Netztopologie bzw. der Schwellenwerte anpassungsfähig\footnote{siehe~\cite[51]{Roj93}} Selbständig zu lernen vermag es nicht.\\


Tatsächlich ist die McCulloch-Pitts-Zelle von eher geringer Bedeutung für die Neurowissenschaft gewesen\footnote{
    ``The immense theoretical influence of this paper was not among neuroscientists but among computer scientists.``~\cite[17]{AR88}; \textit{Rojas} kommt zu einem ähnlichen Schluss: ``Die Art der Schaltungen, die mit
    diesen Zellen gebaut werden, ist aber biologisch gesehen nicht so relevant.``~\cite[51]{Roj93}
}.
Umso größer war ihr Einfluss auf die Computerwissenschaften\footnote{
    Die Arbeit war ebenso wichtig für die Entwicklung des ``Konnektionismus``~\cite[11]{Arb19}, einer Forschungsrichtung der künstlichen Intelligenz, in der Modelle (neuronale Netze) untersucht werden, mit deren Hilfe sich intelligente und kognitive Handlungen auf Maschinen übertragen lassen~\cite[v]{Dor91}
}.
McCulloch und Pitts nahmen für ihr Modell das menschliche Gehirn als Grundlage und ermöglichten so einen neuen Blickwinkel auf Ursache und Wirkung. ``Mind`` no longer ``goes more ghostly than a ghost``~\cite[114]{MP43}:

\blockquote[{\cite[9]{Per88}}]{
    If nerve cells were equivalent to the formal neurons of McCulloch and Pitts and if circuits of such elements could be made arbitrarily complicated, then any kind of animal behavior, however marvelously complicated or however intricately linked to sensory input, could be reproduced, and hence understood, in terms of these circuits.
}

Logische Schaltungen, die Intelligenz und Kognition ermöglichen, brachte den Gedanken an eine ``intelligente Maschine`` hervor (vgl.~\cite[204]{Pic04}): So war ihre Formalisierung ein wichtiges Schlüsselelement für die von-Neumann-Rechnerarchitektur \cite{Neu93} sowie Wieners Kybernetik\footnote{siehe Anhang~\ref{appendix:wiener}}, und zusammen mit den nachfolgenden Arbeiten von \textit{Hebb}\footnote{~\cite{Heb49}} und \textit{Rosenblatt} ebnete es den Weg für die Forschung an künstlicher Intelligenz~\cite[1]{Arb19}.

\noindent
Interessant ist ihr Vermerk, dass der Entwurf von Netzen mit Zyklen ungleich schwerer sei als der für zyklenfreie Netze, denn:

\blockquote[{~\cite[108; Hervorhebung i.O.]{MP43}}]{
    This is largely a consequence of the possibility that activity may be set up in a circuit and continue reverberating around it for an indefinite period of time, so that the realizable \textbf{\textit{Pr}} may involve reference to past events of an indefinite degree of remoteness.
}

\noindent
Eine wie in Gleichung~\ref{eq:gl-z1} beschriebe MCP-Zelle $Z_i(t)$ könnte also in solchen Netzen eine Aktivität $Z_k(t-n)$ referenzieren, bei der $n \in \mathbb{N}$ unbestimmt ist.

\noindent
Besondere Beachtung verdient die Bemerkung von McCulloch und Pitts bzgl. ``Berechenbarkeit``

\blockquote[{~\cite[113]{MP43}}]{
    This is of interest as affording a psychological justification of the Turing definition of computability and its equivalents, Church’s A-definability and Kleene’s primitive recursiveness: if any number can be computed by an organism, it is computable by these definitions, and conversely.
}

\noindent
Sie nehmen hiermit u.a. Bezug auf die von Alan Turing (1912 - 1954) bereits 1936 veröffentlichte Arbeit ``On Computable Numbers, with an Application to the Entscheidungsproblem`` ~\cite{Tur37}, in der Turing die Beschreibung der Operationen seines Computers einleitet mit:

\blockquote[{~\cite[250]{Tur37}}]{
    Let us imagine the operations performed by the computer to be split up into 'simple operations' which are so elementary that it is not easy to imagine them further divided.
}

\noindent
Parallelen zu der Architektur der von Turing aufgestellten Modelle und dem von McCulloch entworfenen ``psychons`` als kleinste Einheit psychischer Aktivitäten sind erkennbar. \textit{Piccinini} stellt darüber hinaus fest, dass die Aussage, MCP-Netze können Berechnungen anstellen, einen ersten wichtigen Bezug zwischen mathematischer Theorie und Neurowissenschaft herstellte~\cite[197]{Pic04}.

\noindent
McCulloch schreibt dazu später

\blockquote[{\cite[164]{Mcc16}}]{
    Pitts and I showed that brains were Turing machines, and that any Turing machine could be made out of neurons.
}

\noindent
und stellt fest, dass das Nervensystem eine ausgezeichnete logische Maschine repräsentiert (vgl. ~\cite[80]{Mcc16}).\\

\noindent
Konsequenterweise nutzt \textit{Minsky} in~\cite[32 ff.]{Min67} die MCP-Zelle als endlichen Automaten\footnote{
    Ein endlicher Automat ist eine informationsverarbeitende Maschine. Basierend auf Eingaben lassen sich mit einer solchen Maschine Zustände, Zustandsübergänge und Ausgaben modellieren. Bei einem \textit{endlichen} Automaten sind die genannten Eingaben, Zustände, und Ausgaben endlich (vgl.~\cite[26 ff.]{SSH95}).
}\footnote{
    dass MCP-Netze endliche Automaten sind: Auch in~\cite[76]{Cow90} sowie~\cite[47, ``Satz 2.4``]{Roj93}. \textit{Arbib} weist darauf hin, dass die Arbeit von McCulloch und Pitts grundlegend für Automatentheorie gewesen ist~\cite[8]{Arb19}
} mit zwei Zuständen, aus denen andere endliche Automaten gebaut werden, und nennt diese, im Jahr 1967 schon geläufig: \textbf{Neuronale Netze}~\cite[33]{Min67}.



\subsection{Der KI-Winter}\label{kiwinter}

In \textit{Perceptrons - An introduction to Computational Geometry (1969)}~\cite{MP88} behandeln \textit{Minsky und Papert} u.a. das Verhalten des Perzeptrons im Fall nicht-separabler Daten\footnote{
    in~\cite[181 ff.]{MP88}
} sowie das Problem bzgl. \textit{recognition of connectedness}\footnote{
    s. \cite[12, ``Theorem 0.8``]{MP88} sowie~\cite[249 f.]{MP88}. In dem Epilog, der 19 Jahre später in einer Neuausgabe von \textit{Minsky und Papert} veröffentlicht wird, machen sie deutlich, dass sie damit nicht nachweisen wollten, dass ein Perzeptron nicht in der Lage zu der Erkennung zusammenhängender geometrischer Figuren sei, sondern dass die Komplexität von verschiedenen Aufgaben Anforderungen an ein Perzeptron stellt, die zu der damaligen Zeit (Ende 1950er / Anfang 1960er) schwierig umzusetzen gewesen sind (vgl. ~\cite[250]{MP88})
}, bei dem es um die Erkennung zusammenhängender geometrischer Figuren geht.
Ihr Anliegen mit dem Buch ist es, die (mathematischen) Grenzen des Rosenblatt-Perzeptrons auf ein formales Gerüst zu stellen (vgl.~\cite[249]{MP88}).
Ihre Ausführungen verstärken aber die ohnehin schon skeptische Haltung\footnote{
    Here was a machine that could do pattern recognition in a humanlike way; it could recognize all kinds of things. Almost everyone at MIT was very skeptical``~\cite[99]{AR98}
} gegenüber der Fähigkeiten des Perzeptrons\footnote{
    auch wegen schlechter Skalierbarkeit des Modells in der Praxis~\cite[159]{AR88}
} und künstlicher neuronaler Netze allgemein, was \textit{Anderson und Rosenfeld} rückblickend auch auf Aussagen in der Einleitung des Buches zurückführen wie

\blockquote[{~\cite[19]{MP88}}]{
    hundreds of projects and experiments [bzgl. des Perzeptrons] were generally disappointing, and the explanations inconclusive. The machines usually work quite well on very simple problems but detoriate very rapidly as the tasks assigned to the get harder.
}

\noindent
Auch eine generelle Irritation und Enttäuschung über das Perzeptron-Modell meinen \textit{Anderson und Rosenfeld} zu erkennen:

\blockquote[{~\cite[232]{MP88}}]{
    we [Minsky und Papert] consider it to be an important research problem to elucidate (or reject) our intuitive judgement that the extension is sterile.
}

\noindent
Diesen ``Pessimismus`` betrachten \textit{Minsky und Papert} im Jahr 1988 in Retrospektive (s. ~\cite[xiii]{MP88}). Sie sind sich der Behauptung bewusst, ihr Buch hätte mit dem Aufzeigen der Grenzen des Rosenblatt-Modells den Forscherdrang an maschinellem Lernen gebremst:

\blockquote[{~\cite[xii]{MP88}}]{
    One popular version is  that the publication of our book so discouraged research on learning in network machines that a promising line of research was interrupted.
}

\noindent
\textit{Anderson und Rosenfeld} erwähnen ``the influence of Marvin Minsky and Seymour Papert on the loss of interest in neural networks during the 1970s``~\cite[X]{AR98}, worüber ihre Interviewpartner in gleicher Quelle sprechen.

\textit{Russell und Norvig} fassen zusammen, dass \textit{Minsky und Papert} in ihrem Buch bewiesen haben, dass ein Perzeptron alles lernen kann, was es auch darstellen kann, aber es könnte halt nur sehr wenig darstellen (vgl.~\cite[45]{RN09})\footnote{
    vgl. hierzu~\cite[xiii]{MP88}
}. Auf gleicher Seite verweisen sie auf die zunehmende Komplexität der zu berechnenden Modelle, die nicht alleine durch schnellere und bessere Hardware kompensiert werden konnte.

Der Lighthill Report\footnote{
    ``Workers entered the field around 1950, and even around 1960, with high hopes that are very far from having been realised in 1972. In no part of the field have the discoveries made so far produced the major impact that was then promised.`` in \url{https://www.chilton-computing.org.uk/inf/literature/reports/lighthill\_report/p001.htm}, ``3 Past disappointments``, abgerufen 28.08.2023
} sollte dann 1973 eine Grundlage für die Entscheidung der britischen Regierung darstellen, das Budget für die Forschung an KI zu kürzen (vgl. ~\cite[45]{RN09}). Neuronale Netze wurden als Grundlage künstlicher Intelligenz zunächst verworfen\footnote{
    vgl.~\cite[641]{Ola96}
}, und die Forschung daran ging zurück, bis sie Anfang/Mitte der 1980er Jahre wieder aufgenommen wurde: Diese Periode ist gemeinhin als ``KI-Winter`` (siehe Anhang~\ref{appendix:kiwinter}) bekannt.





