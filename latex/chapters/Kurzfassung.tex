\kurzfassung

%% deutsch
\paragraph*{}

Neuronale Netze sind in den vergangenen Jahren ein wesentlicher Bestandteil von Forschung und Anwendung in der Medizin geworden.

Die vorliegende Arbeit stellt einige solcher Netze sowie deren theoretischen Unterbauten vor.

Hierzu wird im ersten Teil zunächst die Fachlichkeit zusammengefasst, die Grundlage künstlicher neuronaler Netze ist: Das biologische Neuron mit seinen komplexen biochemischen Vorgängen, die Signalverarbeitung und -weiterleitung in einem Netz aus Nervenzellen ermöglichen.

Darauf aufbauend bietet der zweite Teil eine Einführung in das mathematischen Gerüst des McCulloch-Pitts-Neurons und Rosenblatt-Perzeptrons, zwei frühe Modelle künstlicher Neuronen, die für Forschung und Wissenschaft einen wesentlichen Beitrag geleistet haben.
Anwendungsbeispiele beleuchten die unterschiedlichen Eigenschaften hinsichtlich Statik des McCulloch-Pitts-Neurons und Anpassungsfähigkeit des Rosenblatt-Perzeptrons.

Im dritten Teil werden einige wegweisende Architekturen und Algorithmen künstlicher neuronaler Netze aufgeführt, darunter  Backpropagation sowie Faltungsoperationen, die den tiefen Netzen zu eigen sind, die heutzutage in der Medizin angewendet werden, wo sie grosse Erfolge vorweisen können.

Einige dieser Erfolge werden im vierten Teil vorgestellt.
Ausgewählte Forschungsarbeiten und -Ergebnisse aus dem Gesundheitswesen wie Gesundheitswirtschaft, der Pharmafoschung sowie der Diagnostik und Therapie bezeugen die Leistungsfähigkeit künstlicher neuronaler Netze.
