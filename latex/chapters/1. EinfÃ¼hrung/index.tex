\chapter{Einführung}

Der Einsatz künstlicher neuronaler Netze als Teilgebiet der künstlichen Intelligenz konnte in der Medizin in den letzten Jahren beachtliche Erfolge vorweisen.
Diese Arbeit stellt einige dieser Erfolge vor.
Sie möchte darüber hinaus eine Einführung in Themen und Aspekte aus den Bereichen Neurobiologie, Mathematik und Informatik bieten, um ein erstes Verständnis für Modelle und Verfahren zu schaffen, die zu diesen Erfolgen beigetragen haben.\\

Wir beschäftigen uns deshalb zunächst mit dem Neuron als ``strukturelle und funktionelle Einheit des Nervensystems``~\cite[42]{SD07}.
Im Hinblick auf neuronale Netze ist für uns das natürliche Vorbild der Signalweiterleitung in Nervenzellen von Interesse, weshalb wir uns eingehend mit der Zellmembran und dem \textit{transmembranalen Transport} von Ionen beschäftigen: Es wird sich zeigen, dass Änderungen der Nervenzellmembran-Eigenschaften hinsichtlich ihres Potenzials grundlegend sind für die Ausschüttung von hemmenden oder erregenden Neurotransmittern, die von den Rezeptoren nachgelagerter Nervenzellen weiterverarbeitet und dort gemäß eines Alles-oder-Nichts-Prinzips erneut Signale auslösen können.\\

Dies bildet die Grundlage für die Neuronenmodelle von McCulloch und Pitts sowie Rosenblatt im zweiten Teil der Arbeit, in dem der Schwerpunkt vor allem auf der mathematischen Modellierung von Eingabe- und Ausgabefunktion sowie deren Anwendung liegt: Ein nicht unwesentlicher Teil der zu erarbeitenden Fachlichkeit beschränkt sich also nicht bloß auf die hochkomplexen biochemischen Vorgänge, die in der biologischen Nervenzelle bei Signalverarbeitung und -weiterleitung stattfinden.
Einige diese Vorgänge werden in diesem Abschnitt in Formeln und Gleichungen der Aussagenlogik {bzw.} analytischen Geometrie überführt, damit sie in der Informatik Anwendung finden können.\\

Im dritten und vierten Teil werden darauf aufbauend Architektur und Algorithmen einiger künstlicher neuronaler Netze sowie deren überaus erfolgreiche Anwendung im Gesundheitswesen vorgestellt.
