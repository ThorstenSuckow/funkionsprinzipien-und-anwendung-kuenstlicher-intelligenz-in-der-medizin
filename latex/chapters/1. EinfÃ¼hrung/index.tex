\chapter{Einführung}

Um ein Verständnis für die Strukturen zu erhalten, die ein künstliches neuronales Netz modelliert, wollen wir uns zunächst mit dem Neuron als ``strukturelle und funktionelle Einheit des Nervensystems``~\cite[42]{SD07} beschäftigen.
Von Interesse ist für uns die Funktionsweise dieser Zellen im Kontext von Informationsverarbeitung und -weiterleitung, weshalb wir die molekulare und die zelluläre Ebene des einzelnen Neurons betrachten wollen, sowie den Verbund von Neuronen.

Hierzu skizzieren wir zunächst den Aufbau eines Neurons und verschaffen uns im Anschluss einen Überblick über die komplexen biochemischen Vorgänge, die nötig sind, damit Neuronen Signale senden und empfangen können. Wir werden sehen, dass Änderungen der Zellmembran-Eigenschaften (hier: das Membranpotenzial) eines Neurons mitverantwortlich sind für die Exozytose1 von hemmenden oder erregenden Neurotransmittern, und wie Rezeptoren postsynaptischer Zellen chemische Signale gemäß ihren Eigenschaften zu inhibitorischen2 oder exzitatorischen3 Signalen verarbeiten.

Einige Begrifflichkeiten dieses hochkomplexen Themas, deren voller Hintergrund und Bedeutung, muss im Verborgenen. Das ist dem begrenzten Umfang dieser Arbeit geschuldet. Fußnoten und ausführliche Quellenzitate sollen den interessierten Leser zur eigenständigen Recherche ermutigen.