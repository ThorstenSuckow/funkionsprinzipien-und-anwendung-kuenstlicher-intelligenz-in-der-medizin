\chapter{Einführung}

Um die in dieser Arbeit vorgestellten künstlichen neuronale Netze - sowie deren Erfolge - im Gesundheitswesen besser verstehen zu können, wollen wir zunächst  ihr theoretisches Fundament betrachten.

Im Sinne eines Domain-Modellings wird deshalb zunächst die Fachlichkeit geklärt.
Wir wollen hierdurch ein notwendiges Verständnis für einige Begriffe und Vorgänge schaffen, die uns in der Künstlichen Intelligenz - sobald es um das Thema neuronale Netze geht - begegnen werden. \\

Zunächst beschäftigen wir uns mit dem Neuron als ``strukturelle und funktionelle Einheit des Nervensystems``~\cite[42]{SD07}.
Für uns ist vor allem die Signalweiterleitung von Interesse, weshalb wir uns eingehend mit der Zellmembran und dem \textit{transmembranalen Transport} von Ionen beschäftigen: Es wird sich zeigen, dass Änderungen der Zellmembran-Eigenschaften hinsichtlich des Membranpotenzials grundlegend ist für die Exozytose von hemmenden oder erregenden Neurotransmittern, welche von den Rezeptoren postsynaptischer Zellen als chemische Signale gemäß ihren Eigenschaften zu inhibitorischen oder exzitatorischen Signalen verarbeitet werden.\\

Dies bildet die Grundlage für die Neuronenmodelle von McCulloch und Pitts sowie Rosenblatt im zweiten Teil der Arbeit, in dem der Schwerpunkt vor allem auf der mathematischen Modellierung von Eingabe- und Ausgabefunktion sowie deren Anwendung liegt: Ein nicht unwesentlicher Teil der zu erarbeitenden Fachlichkeit beschränkt sich also nicht bloß auf die hochkomplexen biochemischen Vorgänge, die in der biologischen Nervenzelle bei Signalverarbeitung und -weiterleitung stattfinden.
Ein gewisses grundlegendes mathematisches Verständnis ist vonnöten, wenn diese Vorgänge in diesem Abschnitt in Formeln und Gleichungen der Aussagenlogik {bzw.} analytischen Geometrie überführt werden, damit sie in der Informatik Anwendung finden können\footnote{
    Möglichst verständliche und umfassende Erklärungen zu diesen Themen zu bieten mag wegen des begrenzten Umfangs nicht immer gelingen: Verweise und Anmerkungen in den Fussnoten, Ergänzungen zu Begriffen und Personen im Anhang sowie die zahlreich zitierte Literatur sollen eine weiterführende selbständige Recherche erleichtern.
}.\\

Im dritten und letzten Teil werden darauf aufbauend Architektur und Algorithmen einiger künstlicher neuronaler Netze sowie deren überaus erfolgreiche Anwendung im Gesundheitswesen vorgestellt.