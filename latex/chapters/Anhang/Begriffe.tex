\chapter{Begriffe und Ergänzungen}

\section{Aktionspotenzial}\label{appendix:aktionspotenzial}

\begin{figure}[h]
    \centering
    \includegraphics[
        width=12cm,
        keepaspectratio,
    ]{chapters/2. Das Neuron/images/aktionspotenzial_phasen}
    \caption{Phasen des Aktionspotenzials. (Quelle: in Anlehnung an~\cite[88, Abb. 4.1b]{BCP18})}
  \end{figure}

\subsubsection{Phasen des Aktionspotenzials}
Die Entstehung eines Aktionspotenzials kann entsprechend der zeitlichen Reihenfolge in folgende Phasen eingeteilt werden:

\subsubsection*{1. Aufstrich}
$Na^+$ gelangt in die Zelle,  $g_{Na}$  wird erhöht, mehr $Na+$ strömt ein (positiven Rückkoppelung).
Es kommt zu einer Depolarisation, $V_m$ wächst exponentiell (vgl.~\cite[46]{SD07}).

\subsubsection*{2. Overshoot}
$V_m$ wird positiv ($> 0 mV$) und nähert sich insgesamt $E_{Na} \sim 60 mV$ an\footnote{
    den es nicht erreicht. Erreicht werden Werte zwischen $0$ und $+40 mV$ (vgl.~\cite[69]{FE19})
} (vgl.~\cite[105]{BCP18}).

\subsubsection*{3. Repolarisationsphase}
Auch: Fallende Phase (vgl.~\cite[105]{BCP18}).
Natriumkanäle werden \textbf{inaktiviert}.
Spannungsabhängige Kaliumkanäle öffnen sich ca. $1 ms$ nach Depolarisation\footnote{
    vgl. ~\cite[105]{BCP18} und~\cite[47, Tafel 2.3 (A.2)]{SD07}
}, $K^+$-Ionen strömen aufgrund der elektrochemischen Triebkraft in den EZF, das Membranpotenzial wird wieder negativ\footnote{
    Die Kaliumleitfähigkeit wird ``verzögerter Gleichrichter`` genannt, da sie das ursprüngliche Membranpotenzial - mit Verzögerung - wiederherstellt (vgl.~\cite[103]{BCP18}).
}.


\subsubsection*{4. Undershoot\footnote{
    auch: ``Nachhyperpolarisation``~\cite[46]{SD07}
}}
Es kommt zu einem $V_m$, der unter $V_r$ liegt, da $g_{K}$ noch erhöht ist. $V_m$ nähert sich $E_k$ ($\sim-80 mV$), wozu auch eine erhöhte Pumprate der $Na^+$-$K^+$-ATPase\footnote{
    Die $Na^+$-$K^+$-ATPase ``erhält die Ionenkonzentrationsgradienten aufrecht, die für den Fluss von $Na^+$ und $K^+$ durch die Kanäle während des Aktionspotenzials erforderlich sind``(vgl.~\cite[105]{BCP18}).
} beitragen kann (vgl.~\cite[46]{SD07}).


\subsubsection*{5. Refraktärphase\footnote{
    Die Refraktärphase dient u.a. dazu, die Membran vor einer vorzeitigen Neuerregung zu schützen (vgl.~\cite[76]{Jon19}).
    Hochfrequenten Aktionspotenzialsalven von max. 1000/s sind aufgrund dieser Eigenschaft möglich (vgl.~\cite[46]{SD07} sowie~\cite[89]{BCP18}).
    \textit{Bear et al.} stellen fest: ``Die Frequenz der Aktionspotenziale ist ein Maß für die Stärke des depolarisierenden Stroms``~\cite[89]{BCP18} - je stärker der Reiz, desto mehr Aktionspotenziale werden nacheinander abgefeuert (vgl.~\cite[90, Abb. 4.3]{BCP18}).
}}

- \textit{absolute}: Ca. 2 ms.
nach Auslösen des Aktionspotenzials sind die $Na^+$ Kanäle  \textit{inaktiviert} und dadurch nicht aktivierbar (vgl.~\cite[70]{FE19}).
Es ist keine Aktionspotenzialbildung möglich (vgl.~\cite[46]{SD07}).\\
- \textit{relative}: $V_m$ nähert sich weiter $V_r$ an.
Nachdem einige $Na^+$-Kanäle  \textit{deinaktiviert} wurden, ist eine Auslösung des Aktionspotenzials wieder möglich.
Allerdings ist die Reizschwelle erhöht (da $V_m < V_r < V_t$) und ``die Amplitude des auslösbaren Aktionspotenzials ist reduziert``~\cite[70]{FE19}.

\subsection*{Übertragungsgeschwindigkeit von Signalen in Nervenzellen}
Ist das Axon - die Nervenfaser - \textbf{marklos} , wird das Signal kontinuierlich weitergeleitet, und die Leitungsgeschwindigkeit ist eher gering (ca. $1m/s$).
Die Leitungsgeschwindigkeit hängt hier direkt von dem Durchmesser der Nervenfaser ab und ist proportional zur Wurzel des Fasseradius.
Die marklosen Riesenaxone des Tintenfisches erreichen deshalb aufgrund ihrer Größe Leitungsgeschwindigkeiten von bis zu $20 m/s$ (vgl.~\cite[79]{Jon19}).

Im Gegensatz zu marklosen Nervenfasern erreichen \textbf{myelinisierte Axone} eine höhere Leitungsgeschwindigkeit.
Sie sind gegenüber ihrer Umgebung durch \textbf{Myelinscheiden}\footnote{
    Membranschichten, die sich bis zu 100 mal um das Axon wickeln (vgl.~\cite[79]{Jon19})
} besser isoliert (vgl.~\cite[48]{SD07}).
Dadurch wird der Stromfluss verstärkt und die Erregungsleitung erhöht sich.
Diese Isolierschicht ist ``segmentiert``: Myelinscheiden sind in Abständen von $0,2 - 2mm$ durch sog.  \textit{Ranvier-Schnürringe} unterbrochen.
Deren Membran besitzt spannungsabhängige $Na^+$-Kanäle, die das ankommende elektrische Signal durch  Depolarisierung der einzelnen Segmentabschnitte kontinuierlich weiterleiten - es wird an den Membranabschnitten jeweils ein neues Aktionspotenzial gebildet (vgl.~\cite[48]{SD07}).
Die Weiterleitung in myeliniserten Axonen bezeichnet man deshalb als \textbf{saltatorische} (sprunghafte) Erregungsleitung (vgl.~\cite[109 f.]{BCP18}).

\section{ATPasen}\label{appendix:atpasen}
Kurzform für \textit{Adenosintriphosphasen}: Enzyme, die ATP in ADP und Phosphat aufspalten (vgl.~\cite[26]{SD07}).

\section{Dendriten}\label{appendix:dendriten}
von ``\textgreek{δένδρον} (dendrón)`` (altgriechisch): Baum; einzelne selten länger als $2 mm$ (vgl.~\cite[28]{BCP18}).
Längere Dendriten finden sich an den kortikalen Pyramidenzellen mit einer Länge von $1 cm$ (vgl.~\cite[58]{Eil19}).

\section{Diffusion}\label{appendix:diffusion}
unter der Diffusion (``\textit{diffundere}`` (lat.): zerstreuen, ausbreiten) von Molekülen versteht man ihr Bestreben, entlang eines Konzentrationsgradienten (auch: Konzentrationsgefälles) einen Ausgleich der Konzentrationsunterschiede zu erreichen.
Moleküle in hoher Konzentration diffundieren dann in die Bereiche mit niedriger Konzentration: In den in Abschnitt~\ref{sec-ionenkonzentrationen} betrachteten Beispielen diffundieren bspw. $K^+$-Ionen, bis ein \textit{Gleichgewichtspotenzial} erreicht ist.

\section{Effektoren}
``efficere`` (lat.): bewirken, hervorbringen.
Effektoren können wir uns als Endglied der Signalübertragung vorstellen, auch wenn hier wieder interzelluläre Vorgänge stattfinden. Vgl. ``neuromuskuläre Endplatte`` in ~\cite[127, Abs. 3]{BCP18}.

\section{Expertensysteme}\label{appendix:expertensystem}

1990 von \textit{Friedrich und Stumptner} als ``kommerziell erfolgreichste Teildisziplin der Artificial intelligence`` bezeichnet~\cite[14]{FS90}, und ebenda beschrieben als:

\blockquote[{\cite[14]{FS90}}]{
    Ziel der Expertensysteme ist es, dem Anwender Wissen und Fertigkeiten zur Verfügung zu stellen, über die normalerweise nur speziell ausgebildete oder erfahrene Personen (Experten) verfügen.
}

\noindent
Im groben besteht ein Expertensystem aus einer domänenspezifischen \textit{Wissensbasis}, auf der ein \textit{Inferenzmotor} zum Finden von Antworten und Schlussfolgerungen operiert. \textit{Russel und Norvig} erklären, dass Expertensysteme

\blockquote[{\cite[737]{RN09}}]{
    optimale Entscheidungen empfehlen und dabei die Prioritäten des Benutzers sowie die verfügbaren Evidenzen berücksichtigen
}


\section{Goldman-Gleichung}\label{appendix:goldman}

Auch: \textbf{Goldman-Hodgkin-Katz-Gleichung} (GHK-Gleichung) nach David Eliot Goldman (1910 – 1998), Alan Lloyd Hodgkin(1914 - 1998) und Bernard Katz (1911 - 2003).\\

Wie wir in Abschnitt~\ref{sec-ionenkonzentrationen} gesehen haben, liegt $V_r$ zwischen $-70 mV$ und $- 90mV$. Wie kann man nun schließen, dass das Ruhepotenzial durch die Membranpermeabilität von $K^+$ bestimmt wird, wenn $V_r = -70 mV$, aber $E_K = -80 mV$, und die Membran auch noch für andere Ionen wie bspw. $Na^+$ selektiv permeabel ist\footnote{
    vgl. ~\cite[77]{BCP18} sowie~\cite[44]{SD07}: ``Warum ist $E_m$ weniger negativ als $E_K${?}``
}? \\
Wäre die Membran nur für $K^+$ permeabel, so läge $V_r$ sicher bei $E_k$ (vgl.~\cite[32]{SD07}): Die Nernst-Gleichung kann deshalb nur zur Bestimmung des Membranpotenzials genutzt werden, wenn die Membran nur für ein Ion permeabel ist.
Ansonsten ist der erhaltene Wert nur näherungsweise zu verstehen (vgl.~\cite[67]{FE19}).\\

Das Membranpotenzial kann auch unter Berücksichtigung mehrerer Ionen bestimmt werden: Ionenkanäle unterstützen einen \textit{passiven Transport} der Ionen zwischen EZF und IZF \textit{entlang} ihres Konzentrationsgefälles, während Ionenpumpen, die \textit{entgegen} des Konzentrationsgefälles arbeiten, \textit{aktiv transportieren} (vgl.~\cite[30]{Fro19})\footnote{
    hierfür wird metabolische Energie verbraucht (vgl.~\cite[31]{Fro19}).
}.
Ionenpumpen sind für die Ionenkonzentrationsgradienten und deren Aufrechterhaltung verantwortlich (vgl.~\cite[76]{BCP18})\footnote{
    es wird ein nicht unwesentlicher Teil von Energie zur Aufrechterhaltung dieser Gradienten verbraucht. Die Natrium-Kalium-Pumpe verbraucht laut \textit{Bear et al.} etwa $70$ \% der ATP-Menge, die das Gehirn benötigt (vgl.~\cite[76]{BCP18}).
}.
Um $V_m$ zu berechnen müssen die Ionen mitberücksichtigt werden, für die die Membran permeabel ist.
Dazu kann die \textbf{Goldman-Gleichung} genutzt werden\footnote{
    \textit{Silbernagl und Despopoulos} nutzen für die Bestimmung von $V_m$ die fraktionelle Leitfähgkeit der involvierten Ionen und rechnen $V_r = E_K \times f_K + E_{Na}  \times f_{Na} + E_{Cl}  \times f_{Cl}$ (vgl.~\cite[32, 1.21]{SD07})
}.

\textit{Kandel et al.} stellen hierzu fest, dass eine hohe Konzentration eines einzelnen Ions zusammen mit einer hohen Membranpermeabilität für dieses Ion auch einen größeren Beitrag für $V_r$ leistet:

\blockquote[{\cite[135]{KSJ+13}}]{
    when permeability to one ion is exceptionally high, the Goldman equation reduces to the Nernst equation for that ion.
}

\noindent
\textit{Fakler und Eilers} weisen darauf hin,

\blockquote[{\cite[67]{FE19}}]{
    dass die Permeabilitäten in komplizierter Weise von der Membranspannung und den Ionenkonzentrationen {[...]} abhängen und sich meist nur näherungsweise bestimmen lassen.
}


\section{KI Winter}\label{appendix:kiwinter}

Der Begriff ``KI Winter`` wird in der Literatur mit unterschiedlichen Perioden während der Forschung und Förderung von KI in Zusammenhang gebracht.
In dem Kontext des im Abschnitt~\ref{renaissance} erwähnten Lighthill Reports bezieht sich der Begriff auf die Periode nach 1973:

\blockquote[Artificial intelligence at edinburgh university: A perspective\footnote{\url{https://www.inf.ed.ac.uk/about/AIhistory.html}, abgerufen 31.08.2023}]{
    Lighthill's report provoked a massive loss of confidence in AI by the academic establishment in the UK (and to a lesser extent in the US). It persisted for a decade - the so-called 'AI Winter'.
}

\noindent
\textit{Russell und Norvig} beziehen sich auf einen Zeitraum um/ nach 1988, in dem nach einer Phase von Investitionen in Milliardenhöhe in den Forschungszweig ``viele Unternehmen verschwanden, weil sie ihre außergewöhnlichen Versprechungen nicht halten konnten.``~\cite[48]{RN09}. Auf gleichen Zeitraum bezieht sich \textit{McCorduck} in~\cite[432]{Mcc04}; vgl. hierzu auch:

\blockquote[{\cite[656; Hervorhebung i.O.]{Gar19}}]{
    Dozens of expert systems companies and AI-focused hardware manufacturers failed \textit{en masse} as hype turned to disillusionment.
}



\section{LeNet}\label{appendix:lenet1}

\textit{LeCun et al.} weisen darauf hin, dass LeNet eine Erweiterung der von LeCun beschriebenen Architektur in~\cite{Cun89} ist (vgl.~\cite[544]{CBD+89}).

Spätere Iterationen von \textit{LeNet} durch \textit{LeNet-4} und \textit{LeNet-5}. Die Architektur von \textit{LeNet-5} wird später angemessen mit der in Mitte der 90er Jahre zur Verfügung stehenden Technologie skaliert:

\blockquote[{\cite[15]{CBBH98}}]{
    In 1989 a recognizer as complex as LeNet-5 would have required several weeks' of training, and more data than was available, and was therefore not even considered.
}
\noindent
Eingabedaten für Lenet-5 sind ein $32 \times 32$ Pixel großes Bild, außerdem besitzt das Netz 7 Schichten (ohne Eingabeschicht) (\textit{LeNet-1}: 3 verborgene Schichten (vgl.~\cite[544]{CBD+89})).
Eine vollständige Beschreibung der eindrucksvollen Architektur findet sich in~\cite[7 f.]{CBBH98}

\section{Lernrate}
Die \textbf{Lernrate} ($\eta$,in der Literatur auch $\alpha$) ist der Koeffizient für die Gewichtsanpassungen in einem künstlichen Neuron.

Wie in Abschnitt~\ref{sec:lernregel} gesehen, berechnet die Lernregel die Gewichte auf Basis des \textit{Fehlers}, also der Differenz von $\text{erwartete Ausgabe}$ und $\text{tatsächliche Ausgabe}$: Ist in den Beispielen der Fehler $-1$, werden die Gewichte verringert, ist der Fehler $1$, werden die Gewichte erhöht.
Die Lernrate $\eta$ ist der Koeffizient für die Gewichtsanpassung, und wird auch \textit{Schrittweite} genannt (vgl.~\cite[93]{GBC18} und~\cite[172]{RN09}).

Üblicherweise liegt $\eta$ zwischen $0$ und $1$\footnote{
    vgl.~\cite[61]{Fau94}; außerdem: ``$\alpha$ [$\eta$] ist eine kleine Konstante``~\cite[172]{RN09}. Gleiche Stelle beleuchtet das Für und Wider kleiner und großer Werte von $\eta$; \textit{Salomon} weist in~\cite[173]{Sal90} darauf hin, dass eine geeignete Lernrate auch von der Aufgabenstellung abhängt.
}.


\section{McCulloch-Pitts-Netz als Graph}\label{appendix:mpc-graph}

Gerichtete Kanten eine MCP-Netzes nehmen eigentlich keine Gewichtung der Information vor~\cite[40]{Roj93} (\textit{Rojas} verweist darauf, dass Gewichte in absolut hemmenden Leitungen unsinnig sind: Siehe~\cite[42]{Roj93}). Sowohl \textit{Minsky} in~\cite[34]{Min67} als auch \textit{Rojas} in~\cite[32]{Roj93} nutzen unter Berücksichtigung der \textit{absoluten Hemmung} nur die Summe der erregenden Signale und vergleichen diese mit $\Theta$ - dieser Vergleich findet nur statt, wenn \textit{kein} hemmendes Signal ankommt: Die Zelle liefert sonst direkt $0$ als Ausgabe zurück. \textit{Fausett} nutzt unter Berücksichtigung von Gleichung~\ref{eq:gl-activation} gewichtete Kanten, in ähnlicher Weise auch \textit{Beale und Jackson} in~\cite[41]{BJ90}. \textit{Minsky} weist auf die Äquivalenz eines solchen Modells hin (vgl. ~\cite[34 f.]{Min67}).


\section{Nernst-Gleichung}\label{appendix:nernst-gleichung}
Es gilt, dass sich das Membranpotenzial dem Gleichgewichtspotenzial desjenigen Ions annähert, für das die Membran besonders permeabel ist (vgl. ~\cite[145 f.]{KSJ+13}): Bestimmen lässt sich das Gleichgewichtspotenzial für individuelle Ionen mit der Nernst-Gleichung\footnote{
    vgl.~\cite[67]{FE19}.
} (s. Gleichung~\ref{eq:gl-nernst} sowie Tabelle~\ref{tab:nernstkonstanten}):\\

\textit{Bear et al.} definieren\footnote{siehe ~\cite[74, Exkurs 3.2]{BCP18}}:
\begin{equation}
    E_{Ion} = 2,303  \times \begin{matrix} RT \\ \hline zF \end{matrix} \times log_{10} \times \begin{matrix} [Ion]_{EZF} \\ \hline [Ion]_{IZF} \end{matrix}
    \label{eq:gl-nernst}
\end{equation}



{\renewcommand{\arraystretch}{1.5}%
\begin{table} %[hbtp]
    \begin{center}
        \begin{tabular}{l |l }
            \textbf{Variable / Konstante} & \textbf{Bedeutung}  \\
            \hline
            $E_{ion}$            & Gleichgewichtspotenzial für das jeweilige Ion \\
            $R$                  & Gaskonstante \\
            $T$                  & absolute Temperatur \\
            $z$                  & Ladungszahl des Ions \\
            $F$                  & Faraday-Konstante \\
            $[Ion]_{EZF}$        & Ionenkonzentration \textbf{ausserhalb} der Zelle \\
            $[Ion]_{IZF}$        & Ionenkonzentration \textbf{innerhalb} der Zelle \\
        \end{tabular}
        \caption{Nomenklatur Nernst-Gleichung}
        \label{tab:nernstkonstanten}
    \end{center}
    \small{
        $E$: \textit{Equilibrium} (``Gleichgewicht``); \textit{Faraday-Konstante}: elektrische Ladung eines Mols einfach geladener Ionen; 1 Mol = $6,02214076 \times 10^{23}$ Teilchen

    }

\end{table}

\noindent
Für eine Körpertemperatur von 37° lässt sich die Nernst-Gleichung für das Gleichgewichtspotenzial $E_K$ wie folgt vereinfachen:

\begin{equation}
    E_{K} = 61,54 mV  \times log_{10} \begin{matrix} [K^+]_{EZF} \\ \hline [K^+]_{IZF} \end{matrix}
    \label{eq:gl-nernst-reduced-start}
\end{equation}

\noindent
Mit den Werten aus Tabelle~\ref{tab:ionenkonzentration}  ergibt sich somit


\begin{equation}
    E_{K} = 61,54 mV  \times log_{10} \begin{matrix} 1 \\ \hline 20 \end{matrix} \approx -80 mV
    \label{eq:gl-nernst-reduced-end}
\end{equation}

\noindent

Zur Bestimmung von $V_m$ einer für mehrere Ionen permeablen Membran kann die \textbf{Goldman-Gleichung}\footnote{
    siehe Anhang~\ref{appendix:goldman}
} verwendet werden:

\begin{equation}
    V_{r} = \begin{matrix} RT \\ \hline F \end{matrix}  \times ln \begin{matrix}
                                                                      P_{Na} \space  \times \space [Na^+]_{EZF} \space + \space P_{K} \space  \times \space [K^+]_{EZF} \space + \space P_{Cl} \space  \times \space [Cl^-]_{IZF}  \\ \hline
                                                                      P_{Na} \space  \times [Na^+]_{IZF} \space + \space P_{K} \space  \times \space [K^+]_{IZF} \space + \space P_{Cl} \space  \times \space [Cl^-]_{EZF}
    \end{matrix}
    \label{eq:gl-goldman}
\end{equation}

Eine weitere wichtige Eigenschaft der Membran ist die \textit{elektrische Leitfähigkeit} $g_{Ion}$; für sie gilt, dass sie proportional zu der Anzahl der offenen Ionenkanäle für $Ion$ ist (vgl.~\cite[93]{BCP18}).


\section{Neurotransmitter und ihre Rezeptoren}\label{appendix:neurotransmitter}

Die Diffusion der Vesikel in den synaptischen Spalt dauert ca. $10$ – $100$µs, danach binden sich die Transmitter an die Rezeptoren der postsynaptischen Zelle (vgl.~\cite[96]{HS19a}), die das ``interzellulläre chemische Signal [...] in ein intrazelluläres Signal (eine Änderung des Membranpotenzials oder eine chemische Veränderung) in der postsynaptischen Zelle umwandeln``~\cite[123]{BCP18}.

Rezeptoren können hierbei \textit{ionotrop} oder \textit{metabotrop} sein: Inotrope Rezeptoren sind gleichzeitig auch Ionenkanäle und aktivieren sich, wenn ein bestimmter Transmitter an sie bindet (vgl.~\cite[109]{HS19b}).

Metabotrope Rezeptoren lösen intrazelluläre Stoffwechselvorgänge aus, und weitere Botenstoffe (sog. \textbf{second messenger}) können dann für eine Aktivierung von Ionenkanälen verantwortlich sein (vgl.~\cite[134]{RK18}).

Als Neurotransmitter spielen im zentralen Nervensystem vor allem die Aminosäuren \textit{Glutamat} (Glu, erregend), \textit{Gamma-Aminobuttersäure} (GABA, hemmend)\footnote{
    GABA: Gamma-Aminobuttersäure. Um den GABA-Spiegel zu erhöhen, kann bspw. \textit{Gabapentin} verabreicht werden (\url{https://www.aerzteblatt.de/archiv/20049/Neuropathien-Gabapentin-bremst-ueberaktive-Neurone}, abgerufen 01.08.2023): Es ist bei der Behandlung von Anfallsleiden wie der Epilepsie sowie bei Nervenschmerzen (Neuropathien) wirksam (\url{https://www.gelbe-liste.de/wirkstoffe/Gabapentin_21579}, abgerufen 01.08.2023)
} und Glycin (Gly, hemmend) eine Rolle. An neuromuskulären Endplatten vermittelt Amin \textit{Acetylcholin} (ACh, erregend)\footnote{zur Entdeckung von ACh siehe Anhang~\ref{appendix:loewi}}.

Schnelle Formen der synaptischen Übertragung dauern zwischen $2$ und $100ms$, langsame Übertragungen einige $100$ Millisekunden bis Minuten (vgl.~\cite[129 f.]{BCP18}). Transmitter werden oft zusammen mit Co-Transmittern ausgeschüttet, die die Erregungsübertragung modulieren (vgl.~\cite[52]{SD07}).\\

Ob ein exzitatorischer oder inhibitorischer Transmitter auch dieselbe Wirkung in der postsynaptische Zelle hervorruft, entscheidet sich bei den Rezeptoren(vgl.~\cite[109]{HS19b}). Als Beispiel sei das bereits oben erwähnte ACh genannt, das die Kontraktion des Herzens verlangsamt, bei der Skelettmuskulatur jedoch eine schnelle Depolarisation der Muskelfasern bewirkt (vgl.~\cite[137]{BCP18}).

Erregende und hemmende Synapsen kann man auch an ihrer Struktur erkennen: Asymmetrische oder \textit{Gray-Typ-I-Synapsen} sind auf der postsynaptischen Seite dicker als auf der präsynaptischen, bei gleicher Dimension der Membrandifferenzierungen spricht man von \textit{Gray-Typ-II-Synapsen}. Typ-I sind in der Regel exzitatorisch, Typ-II inhibitorisch (vgl.~\cite[127 u. 147]{BCP18}).


\section{Perzeptron}\label{appendix:perzeptron}
\subsection*{The Perceptron - A perceiving and recognizing automaton}
Das Forschungsprojekt ``\textit{Perceiving and Recognizing Automaton}`` beschreibt einen Apparat, der mittels einer Kamera geometrische Figuren erkennen und zuordnen kann (vgl.~\cite[3]{Ros57}).
Die Funktionen simuliert Rosenblatt zunächst auf einem IBM 704 Rechner\footnote{
    ausführlich beschrieben in~\cite{Ros60}
}, bevor  die Hardware Anfang der 60er Jahre als \textit{Mark 1 Perceptron} gebaut wird: 400 Cadmiumsulfid-Photozellen auf einem 20x20 großen Raster angeordnet - dem \textbf{S-System} (\textbf{S} = \textit{Sensory}) - leiten Signale an das \textbf{A-System} (\textbf{A} = \textit{Association}); dort werden sie registriert und ausgewertet, und schlussendlich über das \textbf{R-System} (\textbf{R} = \textit{Response}) ausgegeben (vgl.~\cite[4 f.]{Ros57},~\cite[389 f.]{Ros58} sowie~\cite[193, ``Frank Rosenblatt]{Bis06} und~\cite[196, Figure 4.8]{Bis06}).
Dabei lernt die Maschine im ersten Schritt durch die Unterstützung der Ingenieure, wie gegebene Formen zu interpretieren sind: Für aktivierte Photozellen wird die erwartete Ausgabe manuell festgelegt.
Die Verbindungen zwischen den \textbf{S}-, \textbf{A}- und \textbf{R}-Units erinnert nicht nur von der Namensgebung her an biologische Neuronen, auch deren Struktur und Verschaltung wird hier als Vorbild genommen (vgl.~\cite[4]{Ros62}).\\

Die \textbf{S}-Units konnten sowohl hemmende als auch erregende Signale in das \textbf{A}-System einspeisen.
Darüber hinaus war das \textbf{R}-System in der Lage, über Rückkoppelungen hemmende Signale an das \textbf{A}-System zu senden: Damit sollte verhindert werden, dass weitere \textbf{R}-Units aktiviert werden, die sich mit den bereits aktivierten Units gegenseitig ausschließen (vgl.~\cite[4 f.]{Ros57}).

Für Aufsehen sorgte das Perzeptron nicht nur im Wissenschaftsbetrieb, sondern auch in der Öffentlichkeit.
In der Presse publikumswirksam, aber wenig vertrauenserweckend: ``Frankenstein Monster Designed by Navy Robot That Thinks``~\cite[v]{Ros62}. Vgl. dazu aktuelle Schlagzeilen der hiesigen Boulevardpresse zum Thema KI. bild.de titelt: ``Übernehmen Computer die Weltherrschaft{?}`` (\url{https://www.bild.de/news/ausland/news-ausland/experten-warnen-ki-so-gefaehrlich-wie-pandemien-und-atomkrieg-84130180.bild.html}, abgerufen 27.08.2023) als Reaktion auf ``Statement on AI Risk`` des \textit{Center for AI Safety}, in dem zahlreiche Wissenschaftler zur Vorsicht beim Umgang, Einsatz und Forschung von KI mahnen.
Im Wortlaut: ``Mitigating the risk of extinction from AI should be a global priority alongside other societal-scale risks such as pandemics and nuclear war.`` (\url{https://www.safe.ai/statement-on-ai-risk}, abgerufen 27.08.2023)


\section{Schwellenwert}\label{appendix:schwellenwert}
Als \textbf{Schwellenwert} wird das Membranpotenzial bezeichnet, bei dem die Permeabilität für $Na^+$ größer als für $K^+$ ist (vgl.~\cite[103]{BCP18})\footnote{$V_r$ ist in Ruhe nahe an $E_K$.}; $V_t$ liegt bei ca. $- 50mV$, unabhängig vom Typ des Neurons (vgl.~\cite[75]{Jon19}).
Damit sich $V_r$ an $V_t$ nähert, bedarf es einer Erregung der Zelle durch postsynaptische Potenziale (vgl. ~\cite[69]{FE19}) oder ``eine aus der Umgebung weitergeleitete (elektrotonische) Erregung``~\cite[46]{SD07}.

Die Stärke der Erregung der Zelle ist entscheidend für das Auslösen eines Aktionspotenzials. Es muss mehr $Na^+$ in die Zelle einströmen, als $K^+$ aus der Zelle ausströmen kann (vgl.~\cite[69]{FE19}).
Versuche zeigen, dass die Potenzialänderung der Membran in einem Bereich von $-80 mV$ zu $-65 mV$ in dieser Hinsicht kaum Änderung bewirkt (vgl.~\cite[99]{BCP18}). $g_{Na}$ erhöht sich, aber wenn $V_r$ nicht erreicht wird, bleibt es bei dieser ``lokalen Antwort``~\cite[46]{SD07}. Erst eine Depolarisation der Membran hin über diesen Wert\footnote{
    ab -60 mV gehen die Natrium-Kanäle in den Offen-Zustand über (vgl.~\cite[69]{FE19}).
} kann die \textbf{Initiationsphase}\footnote{siehe ~\cite[68]{FE19}} des Aktionspotenzials einleiten: Es öffnen sich mehr Natrium-Kanäle, und durch die negative Ladung der Membran-Innenseite gibt es eine starke elektrochemische Triebkraft für $Na^+$-Ionen (vgl.~\cite[103]{BCP18}). Die Triebkraft ist zu diesem Zeitpunkt die Differenz des Membranpotenzials $V_m$ und $E_{Na}$. Bei $-60 mV$ beträgt die Triebkraft\footnote{
    vgl.~\cite[39]{Fak19}
}:

\begin{equation}
    V_m - E_{Na} = -60 mV - 61,54 mV = -121,54 mV
    \label{eq:gl-triebkraft}
\end{equation}


Da sich durch zunehmende Öffnung der Natrium-Kanäle $g_{Na}$ erhöht, strömen aufgrund der hohen Triebkraft für $Na^+$ mehr $Na^+$-Ionen in das Zellinnere. Es kommt zu einem \textbf{Rückkopplungseffekt}, denn die zunehmend weniger negative Membranspannung öffnet weitere Natrium-Kanäle, die Leitfähigkeit der Membran wird weiter erhöht und es kommt zu einem exponentiellen Anstieg der $Na^+$-Konzentration in der IZF. Der ``explosionsartige`` Natriumeinstrom bewirkt die Depolarisation der Membranspannung und das Aktionspotenzial wird ausgelöst (vgl.~\cite[69]{FE19})\footnote{
    siehe hierzu auch Anhang~\ref{appendix:aktionspotenzial}
}.

Die Stärke des Signals selber ist unabhängig von dem Wert, der zu der überschwelligen Reizung des Neurons geführt hat: Amplitude und Zeitverlauf des Signals im Axon hängen nicht von Intensität und Dauer des Reizes ab (vgl.~\cite[75]{Jon19}). Entweder kommt es zu einem Aktionspotenzial, oder es bleibt bei einer lokalen Antwort.

\section{Shunting Inhibition}\label{appendix:shuntinginhibition}
\subsection*{Kurzschlusshemmung bei Neuronen}
Wenn ein Transmitter einer inhibitorischen Synapse $Cl^-$-Kanäle aktiviert, verschiebt sich das Membranpotenzial in Richtung $E_{Cl} \sim -65 mV$.
Ist das Membranpotenzial der Zelle $> -65 mV$, kann ein hyperpolarisierendes IPSP ausgelöst werden.
Wenn allerdings bereits $V_m = E_{Cl}$ ist\footnote{
    ``häufig bei für Chloridleitfähigen GABA Rezeptoren``~\cite[100]{HS19a}
}, wird kein ``sichtbares`` IPSP ausgelöst (vgl. ~\cite[145 f.]{BCP18}) - die $Cl^-$-Kanäle sind offen, aber es findet keine Nettoionenbewegung statt.
Eine Hemmung der Zelle kann aber trotzdem noch stattfinden, wenn bspw. ein distal liegende exzitatorische Synapse positive Ladungsströme verursacht, und die inhibitorische Synapse proximal zum Soma liegt, in Stromrichtung ausgehend von der exzitatorischen Synapse.

Der positive Ladungsstrom kommt dann an der inhibitorischen Synapse mit den geöffnet $Cl^-$-Kanälen an, und die Depolarisation durch das EPSP wird durch den Einfluss von $Cl^-$-Ionen (die $V_m$ wieder auf $V_{rev} = E_{Cl}$ bringen wollen) mitigiert.\\

Aus diesem Grund sind inhibitorische Synapsen oft proximal zum Soma zu finden (vgl.~\cite[231]{KSJ+13}).


\section{Soma}\label{appendix:soma}
von ``\textgreek{σῶμα} (sõma)`` (altgriechisch): Körper; auch \textit{Perikaryon}~\cite[58]{RK18}.
Bezeichnet den Zellkörper und das Stoffwechselzentrum des Neurons mit der eine Größe von ca. $20$ μm~\cite[29]{BCP18}.
Zum Vergleich: ein menschliches Haar hat einen Durchmesser von ca. $70$ μm, kleine Bakterien bis zu $20$ μm.


\section{Synaptische Übertragung und Integration}\label{appendix:synaptische-integration}
\subsection*{Synaptische Übertragung}\label{sec:synaptischeuebertragung}
Neben elektrischen Synapsen, bei denen der Signalaustausch durch einen direkten Stromfluss über sogenannte \textit{gap junctions} und deren ionenleitfähige Verbindungen\footnote{
    ``\textit{Konnexone}``; vgl.~\cite[50]{SD07}
} passiert (vgl.~\cite[119]{BCP18}), erfolgt die Signalweiterleitung im Gehirn überwiegend auf chemische Weise (vgl.~\cite[121 ff.]{BCP18}).

Chemische Synapsen sind nicht direkt miteinander verbunden.
Zwischen ihnen existiert ein Spalt, der ca. $20$ - $40 nm$ breit ist (vgl.~\cite[184]{KSJ+13}), und in der sich eine Matrix aus extrazellulären Proteinen befindet\footnote{
    die \textit{extrazelluläre Matrix}. Hierbei handelt sich um den Gewebeanteil im \textit{Interzellularraum}, also der Raum ausserhalb der Zellen: ``The extracellular matrix {[...]} surrounds all connective tissue cells providing mechanical support and physical strength to tissues, organs and the organism as a whole``~\cite[3]{AHH+98}
}, die den Synapsenspalt überbrückt (vgl. ~\cite[122]{BCP18}).
Die Übertragung von Signalen erfolgt über Exozytose: Botenstoffe (\textit{Neurotransmittern}) diffundieren aus den präsynaptischen Endigungen in diesen Spalt (vgl.~\cite[122]{BCP18}). Rezeptoren an den postsynaptischen Endigungen wandeln die Botenstoffe in hemmende oder erregende Signale um, die dann von der postsynaptischen Zelle nach dem Alles-oder-Nichts-Prinzip integriert werden\footnote{siehe hierzu auch Anhang~\ref{appendix:neurotransmitter}}.\\

Die durch das Aktionspotenzial ausgelöste Depolarisation der Membran an den Axonterminalen bewirkt eine Öffnung spannungsgeladener Calcium-Kanäle (vgl.~\cite[184]{KSJ+13}).
Durch die ungleiche $Ca^{2+}$-Ionenkonzentration zwischen der EZF und IZF (10.000 : 1, siehe Tabelle~\ref{tab:ionenkonzentration}) entsteht eine hohe Triebkraft für $Ca^{2+}$: Nach Gleichung~\ref{eq:gl-nernst} ergibt sich für das Gleichgewichtspotenzial für $Ca^2+$

\begin{equation}
    E_{Ca^{2+}} = 123,08 mV
    \label{eq:gl-eqca2}
\end{equation}


und bei einem Membranpotenzial von $V_m \sim 20 mV$ durch Depolarisation liegt die Triebkraft für $Ca^{2+}$ bei $\sim -100 mV$:

\begin{equation}
    V_m - E_{Ca^{2+}} = 20 mV - 123,08 mV = -103,08 mV
    \label{eq:gl-triebkraftca2}
\end{equation}


Die Calcium-Ionen strömen in das Innere der Zelle und lösen die Exozytose von \textbf{synaptischen Vesikeln}\footnote{
    ``\textit{vesicula}`` (lat.): ``Bläschen``
} aus, kleine, mit einer Membran von der IZF getrennte Strukturen von etwa $50 nm$ Durchmesser, die mit Neurotransmittern gefüllt sind (vgl.~\cite[1000]{BCP18}).
Im \textit{synaptischen Endknöpfchen} befinden sich 100--200 von diesen Bläschen, die jeweils tausende Moleküle eines Neurotransmitters beinhalten (vgl.~\cite[184]{KSJ+13}).

Der Calciumeinstrom in die präsynaptische Endigung löst die Verschmelzung dieser Bläschen mit der Zellmembran des Endknöpfchens an der sogenannten \textbf{aktiven Zone}\footnote{
    \textit{Bear et al.} beschreiben das Aussehen der aktiven Zone als ``ein Feld winziger Pyramiden``~\cite[123]{BCP18}; \textit{Rohkamm und Kerner} erklären die direkte Einwirkung (ionotrop) und indirekte Einwirkung (metabotrop) der Neurotransmitter (siehe~\cite[134]{RK18})
} aus: Ein spezialisierter Abschnitt der Membran, der direkt gegenüber der \textbf{postsynaptischen Dichte} liegt\footnote{
    damit ist der Abschnitt der postsynaptischen Endigung gemeint, in der sich die Rezeptoren befinden (vgl.~\cite[96]{HS19a})
}\footnote{
    Kurz nach Beginn der Transmitterübertragung über den synaptischen Spalt findet die \textbf{Endozytose} statt, ein Recyclingprozess, in dem die individuellen Vesikelmembranen wiederhergestellt und mit Neurotransmitter erneut aufgefüllt werden (vgl.~\cite[133]{BCP18}).
}


\subsection*{Synaptische Integration}

Die Neurotransmitter eines einzelnen Vesikels lösen einen minimalen exzitatorischen oder inhibitorischen postsynaptischen Strom aus (\textbf{mEPSC} bzw. \textbf{mIPSC}, \textit{miniature excitatory / inhibitory postsynaptic current})\footnote{
    EPSCs und IPSCs setzen sich aus einzelnen dieser mEPSC bzw. mIPSC zusammen und bilden die kleinste Einheit der postsynaptischen Ströme, weshalb man sie als \textit{Quanten} bezeichnet~\cite[98]{HS19a}.
    Da EPSPs ein Vielfaches des Quantums sind, ``das die Menge an Transmitter in einem einzigen Vesikel und die Anzahl der postsynaptischen Rezeptoren an der Synapse widerspiegelt``, nennt man sie ``gequantelt``~\cite[142]{BCP18}. Mit Hilfe der \textbf{Quantelungsanalyse} lässt sich die Anzahl der an einer synaptischen Übertragung beteiligten Vesikel bestimmen (vgl. \cite[142]{BCP18}).
}.

Wesentlich für die Entstehung eines neuen Aktionspotenzials in der postsynaptischen Zelle ist die Verrechnung dieser Signale, die räumlich oder zeitlich eintreffen.
Hierbei ist die \textbf{räumliche Summation} die Integration von vielen fast gleichzeitig eintreffenden Signalen mehrerer präsynaptischer Zellen, die sich in der Folgezelle zu einem \textbf{exzitatorischen postsynaptischen Potenzial} (EPSP) aufaddieren (vgl.~\cite[101]{HS19a}).
Unter der \textbf{zeitlichen Summation} versteht man die in gewissen Abständen von ein und derselben Synapse hintereinander eintreffenden EPSPs, die jeweils  das Membranpotenzial für nachfolgende Signale zum Schwellenwert hin verschieben\footnote{
    vgl.~\cite[142]{BCP18} sowie~\cite[101]{HS19a}
}.

Ob der Schwellenwert der postsynaptischen Zelle überschritten werden kann ist auch abhängig von dem Einfluss der \textbf{inhibitorischen Synapsen} auf die Zelle: Inhibitorische ``Eingaben``, die hyperpolarisierend wirken, werden von den exzitatorischen Eingaben subtrahiert\footnote{
    \textit{Bear et al.} stellen in~\cite[146, Exkurs 5.6]{BCP18} die Rolle inhibitorischer Synapsen anschaulich dar.}\footnote{Daneben ist auch der Abstand der Synapsen von der Initiationszone sowie die Eigenschaften der dendritischen Membran zu berücksichtigen. So kann die EPSP-Amplitude kleiner werden, wenn Strom auf dem Weg zu dem Axonhügel durch die Membran verloren geht (vgl.~\cite[142 f.]{BCP18}). \textit{Bear et al.} vergleichen die Dendriten diesbzgl. mit einem ``löchrigen Gartenschlauch`` (vgl.~\cite[143]{BCP18}).
} (vgl.~\cite[225]{KSJ+13}).\\


Das Zusammenspiel zwischen exzitatorischen und inhibitorischen Synapsen wird auch durch das \textbf{Umkehrpotenzial} $V_{rev}$ bestimmt\footnote{
    siehe Anhang~\ref{appendix:umkehrpotenzial}
}: Im Allgemeinen gilt, das an erregenden Synapsen \textbf{unspezifische Kationenkanäle} öffnen, deren Umkehrpotenzial im Bereich von $0mV$ liegt: Hier kommt es zu einer Depolarisation. An hemmenden Synapsen öffnen $Cl^-$- oder $K^+$-Kanäle und es gilt dort $V_{rev} \leq V_r$, wonach es meist zu einer Hyperpolarisation kommt (vgl. ~\cite[100]{HS19a}).\\

Die Verrechnung von EPSP und IPSP (inhibitorisches postsynaptisches Potenzial) erfolgt nicht ausschließlich linear durch Summation - inhibitorische Synapsen können auch für einen ``Kurzschluss`` sorgen und somit ein EPSP um ein Vielfaches verkleinern (vgl.~\cite[477]{Sil10}). Man spricht dann von einer \textit{Kurzschlusshemmung} (\textbf{shunting inhibition})\footnote{
    siehe Anhang~\ref{appendix:shuntinginhibition}
}.



\section{Umkehrpotenzial}\label{appendix:umkehrpotenzial}
Wie wir bei der Entstehung des Aktionspotenzials (siehe Abschnitt~\ref{sec:aktionspotenzial}) gesehen haben, transportieren spannungsgesteuerte Ionenkanäle stets entlang des elektrochemischen Gradienten (vgl.~\cite[39]{Fak19}).
Im Folgenden sei die Membran zur Vereinfachung nur permeabel für ein Ion, es gelte weiterhin $V_m < E_{Ion}$: Die Stromrichtung für das Ion ist einwärts IZF. Ist die Triebkraft positiv wegen $V_m > E_{Ion}$, strömt das Ion auswärts EZF. Wenn allerdings

\begin{equation}
    V_m - E_{Ion} = 0
\end{equation}

dann gilt:

\begin{equation}
    V_m = E_{Ion}
\end{equation}

In diesem Fall entspricht die Membranspannung dem Gleichgewichtspotenzial des Ions - es findet keine Nettoionenbewegung statt.\\

Der Wert für $V_m$, bei dem die Differenz zwischen $V_m$ und $E_{Ion}$ bei $0$ liegt, wird das \textit{Umkehrpotenzial} $V_{rev}$ genannt, weil hier ein Vorzeichenwechsel stattfindet: Je nachdem, in welche Richtung sich $V_m$ ändert, ändert sich auch die Richtung des Stroms.
Für ionenspezifische Kanäle entspricht das \textit{Umkehrpotenzial} ihrem \textit{Gleichgewichtspotenzial} (vgl.~\cite[95 f.]{SBB+13}).\\

Sobald eine Membran permeabel für mehr als ein spezifisches Ion ist, muss nach \textit{Kandel et al.} die relative Leitfähigkeit der Membran für diese Ionen sowie deren Gleichgewichtspotenziale zur Bestimmung des Umkehrpotenzials berücksichtigt werden (vgl.~\cite[196, Box 9--1]{KSJ+13}).
Für die ACh-Rezeptoren an der motorischen Endplatte\footnote{
    chemische Synapse, die Erregungen an eine Muskelfaser weiterleitet (vgl.~\cite[56]{SD07})
}, die für $Na^+$ und $K^+$ permeabel sind, folgt damit, dass die Summe ihrer Ströme am Umkehrpotenzial $V_{rev} = 0$ sein muss:

\begin{equation}
    I_{Ka} + I_{Na} = 0
\end{equation}

Da der Membranstrom $I_{Ion}$ gleich dem Produkt der Membranleitfähigkeit und der elektrochemischen Triebkraft für dieses Ion ist (vgl.~\cite[93]{BCP18}), also

\begin{equation}
    I_{Ion} = g_{Ion}  (V_m - E_{Ion})
\end{equation}

liefert Ersetzen die Gleichung

\begin{equation}
    g_{Na}  (V_m - E_{Na}) = g_{K}  (V_m - E_{K})
\end{equation}

Da am Umkehrpotenzial $V_{rev} = V_m$ gilt, können wir $V_m$ durch $E_{rev}$ ersetzen und danach auflösen, was zur folgenden Gleichung führt (vgl.~\cite[196, Box 9-1]{KSJ+13}):

\begin{equation}
    E_{rev} = \begin{matrix}
                  E_{Na} \space * \space (g_{Na} \space / \space g_{K}) \space + \space E_{K}  \\ \hline
                  (g_{Na} \space / \space g_{K}) \space + 1
    \end{matrix}
\end{equation}

Die ACh-Rezeptoren haben an der Endplatte eine relative Leitfähigkeit für $Na^+:K^+$ von $1,8$, für die Gleichgewichtspotenziale gilt $E_{Na} = +55 mV$ und $E_{K} = -100 mV$; somit folgt nach Einsetzen $E_{rev} = 0mV$ (vgl.~\cite[100]{HS19a}).
Wenn nun das Membranpotenzial vor der Einwirkung von ACh $< 0mV$ ist, bewirkt ein Öffnen der Ionenkanäle einen Strom einwärts IZF, um $V_m$ auf $0$ zu bringen (Depolarisation); im umgekehrten Fall $V_m > 0$ fließen Ionen auswärts zu EZF, und es findet eine Hyperpolarisation statt (vgl. ~\cite[136, Exkurs 5.4]{BCP18}).\\


