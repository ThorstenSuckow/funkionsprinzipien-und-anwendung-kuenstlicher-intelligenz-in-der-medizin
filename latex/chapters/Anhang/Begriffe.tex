\chapter{Begriffe}


\subsection{Expertensysteme}\label{appendix:expertensystem}

1990 in [FS90, S. 14, ``1.4 Expertensysteme``] als ``kommerziell erfolgreichste Teildisziplin der Artificial intelligence`` bezeichnet, und ebenda beschrieben als:

\blockquote[{[FS90, S. 14, ``1.4 Expertensysteme``]}]{
    Ziel der Expertensysteme ist es, dem Anwender Wissen und Fertigkeiten zur Verfügung zu stellen, über die normalerweise nur speziell ausgebildete oder erfahrene Personen (Experten) verfügen.
}

\noindent
Im groben besteht ein Expertensystem aus einer domänenspezifischen \textit{Wissensbasis}, auf der ein \textit{Inferenzmotor} zur Findung von Antworten und Schlussfolgerungen operiert. \textit{Russel und Norvig} erklären, dass Expertensysteme

\blockquote[{[RN09, S. 737]}]{
    optimale Entscheidungen empfehlen und dabei die Prioritäten des Benutzers sowie die verfügbaren Evidenzen berücksichtigen
}



\subsection{KI Winter}\label{appendix:kiwinter}

Der Begriff ``KI Winter`` wird in der Literatur mit unterschiedlichen Perioden während der Forschung und Förderung von KI in Zusammenhang gebracht.
In dem Kontext des im Abschnitt~\ref{kiwinter} erwähnten Lighthill Reports bezieht sich der Begriff auf die Periode nach 1973

\blockquote[Artificial intelligence at edinburgh university: A perspective\footnote{https://www.inf.ed.ac.uk/about/AIhistory.html, abgerufen 31.08.2023}]{
    Lighthill's report provoked a massive loss of confidence in AI by the academic establishment in the UK (and to a lesser extent in the US). It persisted for a decade - the so-called 'AI Winter'.
}

\noindent
\textit{Russell und Norvig} beziehen sich auf einen Zeitraum um/ nach 1988, in dem nach einer Phase von Investitionen in Milliardenhöhe in den Forschungszweig ``viele Unternehmen verschwanden, weil sie ihre außergewöhnlichen Versprechungen nicht halten konnten.`` [RN09, S. 48, 1.3.6 letzter Abs.]. Auf gleichen Zeitraum bezieht sich [Mcc04, S. 432 ff.]; vgl. hierzu auch [Gar19, S. 656]:

\blockquote[{[Gar19, S. 656, Hervorhebungen i.O.]}]{
    Dozens of expert systems companies and AI-focused hardware manufacturers failed \textit{en masse} as hype turned to disillusionment.
}



\subsection{LeNet-1}\label{appendix:lenet1}

\textit{LeCun et al.} weisen darauf hin, dass

\blockquote[{[CBD+89 S. 544]}]{
    Its architecture is a direct extension of the one proposed in LeCun (1989)
}
\noindent
Spätere Iterationen von \textit{LeNet} durch \textit{LeNet-4} und \textit{LeNet-5}. Die Architektur von \textit{LeNet-5} wird später angemessen mit der in Mitte der 90er Jahre zur Verfügung stehenden Technologie skaliert:

\blockquote[{[CBB+98:15]}]{
    In 1989 a recognizer as complex as LeNet-5 would have required several weeks' of training, and more data than was available, and was therefore not even considered.
}
\noindent
Eingabedaten für Lenet-5 sind ein $32 \times 32$ Pixel grosses Bild, ausserdem besitzt das Netz 7 Schichten (ohne Eingabeschicht) (\textit{LeNet-1}: 3 verborgene Schichten [CBD+89:544]).
Eine vollständige Beschreibung der eindrucksvollen Architektur findet sich in [CBB+98:7 ff.]


