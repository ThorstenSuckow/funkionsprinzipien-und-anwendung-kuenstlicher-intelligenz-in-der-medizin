\chapter{Personen}

\subsection{Julius Bernstein}\label{appendix:bernstein}

\subsection*{Die Entdeckung bioelektrischer Ströme}

Im Jahr 1912 veröffentlichte der deutsche Physiologe \textit{Julius Bernstein} (1839-1917) Erkenntnisse seiner elektrophysiologischen Studien in dem Buch ``\textit{Elektrobiologie - Die Lehre von den elektrischen Vorgängen im Organismus auf moderner Grundlage dargestellt}``.
In dem Kapitel ``Membrantheorie`` [Ber12:87] greift er das Postulat seiner Arbeit ``Untersuchungen zur Thermodynamik der bioelektrischen Ströme`` [Ber02] auf.
Bereits 1902 von ihm veröffentlicht, lieferte die Theorie die erste plausible physikalisch-chemische Erklärung bioelektrischer Phänomene [Sey06:5. ``4 Membrane Theory (1902)``]:
In Zellen befindet sich eine elektrolytische Flüssigkeit, die von einer für bestimmte Ionen permeablen Membran umgeben ist. Diese Membran besitzt eine Potenzialdifferenz (vgl. {[Ber12:92 f.]})

\blockquote[{[Ber02:542]}]{
    Denken wir uns, dass diese Elektrolyte aus dem Querschnitt der Fibrillen ungehindert in die umgebende Flüssigkeit diffundieren, wahrend sie am Längsschnitt durch die lebende Sarkoplasmahaut daran gehindert werden, well sie für ein Ion derselben, z. B. für das Anion ($PO^-_4$ u. s. w.), mehr oder weniger impermeabel ist, so entstünde auf der Oberfläche der Fibrille eine elektrische Doppelschicht, welche nach innen negative, nach aussen positive Spannung besitzen würde.
}

Knapp 34 Jahre vor seiner Membrantheorie vermutet Bernstein bereits die Existenz eines Aktionspotenzials [Sch83:168]. Seine Forschungsarbeit über elektrische Muskel-Nerven-Aktivitäten {[Ber68]} schließt er mit den Worten:

\blockquote[{[Ber68:198 f.]}]{
    Ich werde im Folgenden diese Welle mit dem Namen 'Reizwelle' bezeichnen, well derjenige Reiz, durch den die Nervenfaser im Centrum Empfindung, im Muskel Zuckung erzeugt, auf das  Innigste mit dieser Welle verknüpft ist. {[...]} Da wir nachgewiesen haben, dass die Reizwelle mit derselben Geschwindigkeit sich fortpflanzt als die Erregung, so können wir die wohlberechtigte Annahme machen, dass die Reizwelle Nichts anderes ist als das Bild des im Nerven ablaufenden Erregungsvorganges.
}

In den darauffolgenden Jahren und Jahrzehnten haben zahlreiche Forschungen die Theorie bestätigt: Wir wissen heute, dass die ungleiche Ionenverteilung in der intrazellulären und extrazellulären Flüssigkeit zusammen mit der Permeabilität der Membran für die Entstehung eines Membranpontenzials verantwortlich ist.

\subsection{Alan Hodgkin und Andrew Huxley}\label{appendix:hodgkinhuxley}

Für die Erforschungen der Ionen-Mechanismen, die bei der Erregung und Hemmung von Nervenzellmembranen beteiligt sind, erhalten Alan L. Hodgkin (1914 - 1998) und Andrew F. Huxley (1917 - 2012) im Jahr 1963 den Nobelpreis für Physiologie oder Medizin, zusammen mit John Carew Eccles (1903 – 1997) {[DMW63]}.

Knapp 80 Jahre nach Bernsteins Beschreibung der ``negativen Schwankungen`` und 50 Jahre nach seiner Membrantheorie führen Hodgkin und Huxley Experimente an Riesenaxonen von Tintenfischen durch, die zu einem mathematischen Modell des Membranverhaltens führt.

Einige Jahre vorher - 1939 - können Curtis und Cole bei ähnlichen Versuchsanordnungen während der Untersuchung von Membraneigenschaften in Folge eines Aktionspotenzials eines sprunghaften Anstieg der Leitfähigkeit der Membran feststellen [CC39:669, Summary].

Die Experimente wurden an Riesenaxomen von Tintenfischen durchgeführt, die sich für die damals zur Verfügung stehende Technik mit einer Dicke von ca. 1 mm besser eigneten als die im direkten Vergleich geradezu unfassbar winzigen Axone von menschlichen Nervenzellen (Durchmesser ca 1µm {[Jon19:79, Abs. 1]}):

\blockquote[{[CC39:650, Abs. 2]}]{
    The large diameter of the axon, 0.5 mm or more, makes it particularly favorable material {[...]}
}

Hodgkin und Huxley weisen später nach, dass der Aufstrich des Aktionspotenzials mit einer Zunahme von $g_{Na}$ und dem Einstrom von $Na^+$ zusammenhängt, und weiter, dass die Repolarisation der Zellmembran mit einer Zunahme von $g_K$ und einem Ausstrom von $K^+$ zusammenhängt\footnote{
    vgl. {[Jon19:75, ``Permeabilitäten``]}, ausserdem {[BCP18:96, ``Das Aktionspotenzial in der Realität``]} sowie {[HH52:530 Fig. 17.]}
}.
Sie stellen Hypothesen u.a. zu der Refraktärzeit und dem Schwellenwert auf:\\

Zum Schwellenwert:
\blockquote[{[HH52:535, ``Threshold``]}]{
    The curves in Figs. 12 and 21 show that the theoretical 'membrane' has a definite threshold when stimulated by a sudden displacement of membrane potential.
}

Zur Refraktärzeit:
\blockquote[{[HH52:532, ``Refractory Period``]}]{
    Acording to our theory, there are two changes resulting from the depolarization during a spike which make the membrane unable to respond to another stimulus until a certain time has elapsed. These are 'inactivation', which reduces the level to which the sodium conductance can be raised by a depolarization, and the delayed rise in potassium conductance, which tends to hold the membrane potential near to the equilibrium value for potassium ions.
}

Ausserdem formulieren ein mathematisches Modell auf Basis des beobachteten Membranverhaltens [HH52:501, Fig. 1.1], heute bekannt als das \textbf{Hodgkin-Huxley-Modell} [Koc99:142 ff.].
 Die Relevanz des Modells fassen \textit{Kandel et al.} so zusammen:

\blockquote[{[KSJ+13:156, Abs. 2]}]{
    More than a half-century later the Hodgkin-Huxley model stands as the most successful quantitative model in neural science if not in all of biology.
}



\subsection{David Hubel, Torsten Wiesel}\label{appendix:hubelwiesel}

David H. Hubel (1926 - 2013) und Torsten N. Wiesel erhalten 1981 den Nobelpreis für Physiologie oder Medizin  für ihre Entdeckungen bzgl. Informationsverarbeitung im Seewahrnehmungssystem\footnotetext{
    https://www.nobelprize.org/prizes/medicine/1981/summary/, abgerufen 05.09.2023
}
In [HW62] (1962) zeigen sie, dass im Wahrnehmungssystem ``einfache`` und ``komplexe`` Neuronen existieren, die visuelle Informationen unterschiedlich verarbeiten (vgl. [Wur09:2819])

\subsection{Walther Nernst}
\textit{Walther Nernst} (1864 - 1941) war ein deutscher Physiker und Chemiker und gehört zu den Begründern der physikalischen Chemie. Er formulierte das \textbf{Nernst-Theorem}, auch bekannt als \textbf{Dritter Hauptsatz der Thermodynamik}. 1920 erhielt er den Chemie-Nobelpreis in Anerkennung seiner Arbeit auf dem Gebiet der Thermochemie.


\subsection{Otto Loewi}
\subsection*{Die Entdeckung des Vagusstoff}

Otto Loewi (1873 - 1961) erhielt 1936 den Nobelpreis für Medizin für seine Forschungen an der chemischen Übertragung von Nervenimpulsen.

1921 konnte er bei einem Experiment mit Froschpräparaten zeigen, daß die Infusionslösung, die für vorher am Vagusnerv bewusst stimulierte Froschherzen genutzt wurde, den Vagusnerv von nachträglich mit dieser Lösung behandelte Froschherzen stimulieren konnte.

Loewi vermutete in der Lösung eine Substanz, die er ``Vagusstoff`` nannte.
Henry Dale (1875 - 1968)- mit dem sich Loewi den Nobelpreis teilte - konnte diesen Stoff später als \textit{Acetylcholin} identifizieren, ein exzitatorischer Neurotransmitter\footnotemark{
    ``[... ]Acetylcholin wirkt erregend an der motorischen Endplatte, aber hemmend an den Schrittmacherzellen des Herzens`` {[HS19c:105, ``Postsynaptische Rezeptoren vermitteln den Transmittereffekt``]}
}, der bei chemischen Synapsen als Botenstoff zur Signalübertragung beteiligt ist\footnote{
    vgl. {[AHH+98]} sowie {[BCP18:119, Exkurs 5.1]}
}.
