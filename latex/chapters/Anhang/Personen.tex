\chapter{Personen}

\section{Julius Bernstein}\label{appendix:bernstein}

\subsection*{Die Entdeckung bioelektrischer Ströme}

Im Jahr 1912 veröffentlichte der deutsche Physiologe \textit{Julius Bernstein} (1839-1917) Erkenntnisse seiner elektrophysiologischen Studien in dem Buch ``\textit{Elektrobiologie - Die Lehre von den elektrischen Vorgängen im Organismus auf moderner Grundlage dargestellt}``.
In dem Kapitel ``Membrantheorie``~\cite{Ber12} greift er das Postulat seiner Arbeit ``Untersuchungen zur Thermodynamik der bioelektrischen Ströme`` ~\cite{Ber02} auf.
Bereits 1902 von ihm veröffentlicht, lieferte die Theorie die erste plausible physikalisch-chemische Erklärung bioelektrischer Phänomene [Sey06:5. ``4 Membrane Theory (1902)``]:
In Zellen befindet sich eine elektrolytische Flüssigkeit, die von einer für bestimmte Ionen permeablen Membran umgeben ist. Diese Membran besitzt eine Potenzialdifferenz (vgl.~\cite[92 f.]{Ber12})

\blockquote[{\cite[542]{Ber02}}]{
    Denken wir uns, dass diese Elektrolyte aus dem Querschnitt der Fibrillen ungehindert in die umgebende Flüssigkeit diffundieren, wahrend sie am Längsschnitt durch die lebende Sarkoplasmahaut daran gehindert werden, well sie für ein Ion derselben, z. B. für das Anion ($PO^-_4$ u. s. w.), mehr oder weniger impermeabel ist, so entstünde auf der Oberfläche der Fibrille eine elektrische Doppelschicht, welche nach innen negative, nach aussen positive Spannung besitzen würde.
}

Knapp 34 Jahre vor seiner Membrantheorie vermutet Bernstein bereits die Existenz eines Aktionspotenzials [Sch83:168]. Seine Forschungsarbeit über elektrische Muskel-Nerven-Aktivitäten~\cite{Ber68} schließt er mit den Worten:

\blockquote[{\cite[198 f.]{Ber68}}]{
    Ich werde im Folgenden diese Welle mit dem Namen 'Reizwelle' bezeichnen, well derjenige Reiz, durch den die Nervenfaser im Centrum Empfindung, im Muskel Zuckung erzeugt, auf das  Innigste mit dieser Welle verknüpft ist. {[...]} Da wir nachgewiesen haben, dass die Reizwelle mit derselben Geschwindigkeit sich fortpflanzt als die Erregung, so können wir die wohlberechtigte Annahme machen, dass die Reizwelle Nichts anderes ist als das Bild des im Nerven ablaufenden Erregungsvorganges.
}

In den darauffolgenden Jahren und Jahrzehnten haben zahlreiche Forschungen die Theorie bestätigt: Wir wissen heute, dass die ungleiche Ionenverteilung in der intrazellulären und extrazellulären Flüssigkeit zusammen mit der Permeabilität der Membran für die Entstehung eines Membranpontenzials verantwortlich ist.

\section{Alan Hodgkin und Andrew Huxley}\label{appendix:hodgkinhuxley}

Für die Erforschungen der Ionen-Mechanismen, die bei der Erregung und Hemmung von Nervenzellmembranen beteiligt sind, erhalten Alan L. Hodgkin (1914 - 1998) und Andrew F. Huxley (1917 - 2012) im Jahr 1963 den Nobelpreis für Physiologie oder Medizin, zusammen mit John Carew Eccles (1903 – 1997)~\cite{Gle09}.

Knapp 80 Jahre nach Bernsteins Beschreibung der ``negativen Schwankungen`` und 50 Jahre nach seiner Membrantheorie führen Hodgkin und Huxley Experimente an Riesenaxonen von Tintenfischen durch, die zu einem mathematischen Modell des Membranverhaltens führt.

Einige Jahre vorher - 1939 - können Curtis und Cole bei ähnlichen Versuchsanordnungen während der Untersuchung von Membraneigenschaften in Folge eines Aktionspotenzials eines sprunghaften Anstieg der Leitfähigkeit der Membran feststellen~\cite[669]{CC39}.

Die Experimente wurden an Riesenaxomen von Tintenfischen durchgeführt, die sich für die damals zur Verfügung stehende Technik mit einer Dicke von ca. 1 mm besser eigneten als die im direkten Vergleich geradezu unfassbar winzigen Axone von menschlichen Nervenzellen (Durchmesser ca 1 µm~\cite[79]{Jon19}):

\blockquote[{\cite[650]{CC39}}]{
    The large diameter of the axon, 0.5 mm or more, makes it particularly favorable material {[...]}
}

Hodgkin und Huxley weisen später nach, dass der Aufstrich des Aktionspotenzials mit einer Zunahme von $g_{Na}$ und dem Einstrom von $Na^+$ zusammenhängt, und weiter, dass die Repolarisation der Zellmembran mit einer Zunahme von $g_K$ und einem Ausstrom von $K^+$ zusammenhängt\footnote{
    vgl.~\cite[75]{Jon19}, ausserdem~\cite[96]{BCP18} sowie~\cite[530 Fig. 17]{HH52}
}.
Sie stellen Hypothesen unter anderem zu der Refraktärzeit und dem Schwellenwert auf:\\

Zum Schwellenwert:
\blockquote[{~\cite[535]{HH52}}]{
    The curves in Figs. 12 and 21 show that the theoretical 'membrane' has a definite threshold when stimulated by a sudden displacement of membrane potential.
}

Zur Refraktärzeit:
\blockquote[{~\cite[532]{HH52}}]{
    Acording to our theory, there are two changes resulting from the depolarization during a spike which make the membrane unable to respond to another stimulus until a certain time has elapsed. These are 'inactivation', which reduces the level to which the sodium conductance can be raised by a depolarization, and the delayed rise in potassium conductance, which tends to hold the membrane potential near to the equilibrium value for potassium ions.
}

Ausserdem formulieren sie ein mathematisches Modell auf Basis des beobachteten Membranverhaltens~\cite[501, Fig. 1.]{HH52}, heute bekannt als das \textbf{Hodgkin-Huxley-Modell}~\cite[144]{Koc98}.
 Die Relevanz des Modells fassen \textit{Kandel et al.} so zusammen:

\blockquote[{\cite[156]{KSJ+13}}]{
    More than a half-century later the Hodgkin-Huxley model stands as the most successful quantitative model in neural science if not in all of biology.
}

\section{Donald Hebb}\label{appendix:hebb}
\subsection*{Cells that fire together, wire together: Synaptische Plastizität}

Donald Hebb (1904 - 1985), gebürtiger Kanadier, Sohn eines Ärzte-Elternpaares und 1965 nominiert für den Nobelpreis~\cite[1013 f.]{BM03}, geht als junger Mann einer Karriere als Schriftsteller nach.
Er studiert Englisch als Hauptfach und macht 1924 seinen Bachelor\footnote{
    an der Dalhousie Universität: \url{https://dal.ca} (abgerufen 17.08.2023), vgl.~\cite[852]{Coo05}
}, doch die Schriften Freuds, mit denen er sich nach seinem Abschluss beschäftigt, wecken in ihm den Wunsch, sich tiefer mit Psychologie zu beschäftigen~\cite[1013]{BM03}: An der McGill Universität in Montreal\footnote{
    McGillUniversität, Montreal, Quebec (Kanada): \url{https://www.mcgill.ca/neuro/about/donald-hebb-phd} (abgerufen 16.08.2023)
} macht er 1932 seinen Master darin\footnote{
    zu der Zeit studierte er in Teilzeit an der McGill Universität: ``as a part time graduate student`` \cite[1]{Kle99}. Seine Master-Arbeit schrieb er aufgrund einer Erkrankung im Bett~\cite[1014]{BM03}
}, und leitet dort 16 Jahre später als Professor die Fakultät für Psychologie~\cite[853]{Coo05}.

Seine Faszination darüber, wie das Gehirn lernt, Informationen verarbeitet und speichert [Str01:298 ff.] wird Bestandteil seiner Forschungsarbeit: 1949 veröffentlicht er das Buch ``\textbf{The organization of Behavior: A Neuropsychological Theory}``~\cite{Heb49}; seine darin formulierten Postulate\footnote{
    es sind ``three pivotal postulates``~\cite[2]{Kle99}
} liefern einen wichtigen Beitrag für die Neurowissenschaften. Als hätte sein Buch eine Art Golgräberstimmung in der der Psychologie ausgelöst, schreibt _Klein_ hierzu:

\blockquote[{~\cite[2]{Kle99}}]{
    ``It attracted many
    brilliant scientists into psychology, made McGill University a North American mecca for scientists interested in brain mechanisms of behaviour, led to many important discoveries, and steered contemporary psychology onto a more fruitful path.``~\cite[1]{Kle99}
}

Oft zitiert wird seine Idee bzgl. synaptischer Verstärkung, was heute als \textbf{Hebbsche Synapse} bekannt ist\footnote{
    s.~\cite[43]{AR88} sowie als Zitat: \textit{a:} ``[...] any two cells or systems of cells that are repeatedly active at the same time will tend to become “associated,” so that activity in one facilitates activity in the other.``~\cite[52]{Heb88} sowie \textit{b:} ``A series of such events [Aktivierung von cell-assemblies] constitutes a “phase sequence”—the thought process``.~\cite[48]{Heb88}; die Aussage findet sich im Original bereits in ~\cite[xi-xix, ``Introduction``]{Heb49}
}:

\blockquote[{~\cite[50, Hervorhebung i.O.]{Heb88}}]{
    \textit{When an axion of cell} A \textit{is near enough to excite a cell} B \textit{and repeatedly or persistently takes part in firing it, some growth process or metabolic change takes place in one or both cells such that} A\textit{'s efficiency, as one of the cells firing} B\textit{, is increased.}
}


Derartige Veränderungen synaptischer Verbindungen wird als \textbf{Hebbsche Lernregel}~\cite[985]{BCP18} bezeichnet.
Das damit verbundene geflügelte Wort \textit{Cells that fire together, wire together}\footnote{
    Zumindest in~\cite{Heb49} soll sich kein solches Zitat finden. \textit{Keysers und Gazzola} stellen hierzu fest: ``This mnemonic phrase was first introduced by Carla Shatz in an article for the Scientific American aimed at lay public``~\cite[2, Fussnotenmarker entfernt]{KG14} und meint damit den Satz ``In a sense, then, cells that fire together wire together.`` in [Sha92:94]. \url{https://en.wikipedia.org/wiki/Hebbian{\_}theory{\#}cite{\_}ref-2} (abgerufen 16.08.2023) hingegen schreibt den Ursprung \textit{Löwel und Singer} zu: ``neurons wire together if they fire together.``~\cite[211]{LS92}
} beschreibt die Hypothese bildhaft.
Seine Idee der ``\textbf{Cell Assembly}`` schließt daran an: Damit sind Verbände von Neuronen gemeint, die miteinander verschaltet sind, und deren Verbindungen durch das Hebbsche Lernen so sehr verstärkt sind, das die Aktivierung einzelner Zellen in diesen Verbänden ausreicht, das alle Zellen aktiviert werden~\cite[907 f.]{BCP18}.


Hebbs Theorien gelten durch die Forschung als bestätigt\footnote{
vgl.~\cite[833]{Flo19}. \textit{Bear et al.} verweisen in~\cite[875, Exkurs 23.5]{BCP18} verweist auf~\cite{CL78}
}, und mit der von Hebb formulierten synaptische Plastizität wurde auch eine Idee für lernende künstliche neuronale Netze geliefert\footnote{
``[Hebb] laid the foundation for neoconnectionism which seeks to explain cognitive processes in terms of connections between assemblies of real or artificial neurons.``~\cite[2]{Kle99}
}.


\section{David Hubel, Torsten Wiesel}\label{appendix:hubelwiesel}

David H. Hubel (1926 - 2013) und Torsten N. Wiesel erhalten 1981 den Nobelpreis für Physiologie oder Medizin  für ihre Entdeckungen bzgl. Informationsverarbeitung im Seewahrnehmungssystem\footnote{
    https://www.nobelprize.org/prizes/medicine/1981/summary/, abgerufen 05.09.2023
}
In~\cite{HW62} (1962) zeigen sie, dass im Wahrnehmungssystem ``einfache`` und ``komplexe`` Neuronen existieren, die visuelle Informationen unterschiedlich verarbeiten (vgl. [Wur09:2819])

\section{Walther Nernst}
\textit{Walther Nernst} (1864 - 1941) war ein deutscher Physiker und Chemiker und gehört zu den Begründern der physikalischen Chemie. Er formulierte das \textbf{Nernst-Theorem}, auch bekannt als \textbf{Dritter Hauptsatz der Thermodynamik}. 1920 erhielt er den Chemie-Nobelpreis in Anerkennung seiner Arbeit auf dem Gebiet der Thermochemie.


\section{Otto Loewi}
\subsection*{Die Entdeckung des Vagusstoff}

Otto Loewi (1873 - 1961) erhielt 1936 den Nobelpreis für Medizin für seine Forschungen an der chemischen Übertragung von Nervenimpulsen.

1921 konnte er bei einem Experiment mit Froschpräparaten zeigen, daß die Infusionslösung, die für vorher am Vagusnerv bewusst stimulierte Froschherzen genutzt wurde, den Vagusnerv von nachträglich mit dieser Lösung behandelte Froschherzen stimulieren konnte.

Loewi vermutete in der Lösung eine Substanz, die er ``Vagusstoff`` nannte.
Henry Dale (1875 - 1968)- mit dem sich Loewi den Nobelpreis teilte - konnte diesen Stoff später als \textit{Acetylcholin} identifizieren, ein exzitatorischer Neurotransmitter\footnote{
    ``[... ]Acetylcholin wirkt erregend an der motorischen Endplatte, aber hemmend an den Schrittmacherzellen des Herzens``~\cite[105]{HS19b}.
}, der bei chemischen Synapsen als Botenstoff zur Signalübertragung beteiligt ist\footnote{
    vgl.~\cite[119, Exkurs 5.1]{BCP18}
}.

\section{Warren McCulloch und Walter Pitts}\label{appendix:mcculloch}
\subsection*{Von der Philosophie zur Neurowissenschaft}

Warren McCulloch war studierter Pyschologe, Philosoph und Neurophysiologe.
Schon früh in seiner Karriere\footnote{
    die im folgenden zitierte Frage wird von der Quelle auf das Jahr 1917 datiert, McCullochs erstem College-Jahr in Haverford, Pennsylvania
} trieben ihn grundlegende philosophische Fragen um~\cite[1]{Arb19}:

\blockquote[{\cite[2]{Mcc16}}]{
    What is a number, that a man may know it, and a man that he may know a number{?}
}

\noindent
``What is a number`` beantwortet ihm die Mathematik; der zweite Teil der Frage soll ihn Zeit seines Lebens beschäftigen und Bestandteil seiner Arbeit werden~\cite[4 f.]{Abr02}.
Die Suche nach Antworten auf solche Fragen\footnote{
    ``In addition to the work in functional neuroanatomy, McCulloch was continuing to pursue philosophical questions, such as 'What could be the logic of the brain?'``~\cite[3]{Arb19}
} drängt ihn in die Neurowissenschaften\footnote{
``The Logical Structure of Mind: An Inquiry into the Philosophical Foundation of Psychology and Psychiatry``, S. iii;\url{https://ntrs.nasa.gov/citations/19650017787}, abgerufen 19.09.2023

}.
Um psychische Ereignisse auf das wesentlichste zu reduzieren und besser erklärbar zu machen, formuliert er das \textit{``psychon``}, ein einzelnes psychisches Ereignis, das u.a. folgende Eigenschaften hat:

\begin{enumerate}
    \item Es tritt auf, oder es tritt nicht auf
    \item Wenn es auftritt, dann nur, weil es von einem vorhergehenden Ereignis augelöst wurde
    \item Es soll ein nachgehendes ``psychon`` auslösen können\footnote{
        Charakteristika der zweiwertigen Aussagenlogik sind hier bereits erkennbar (vgl. ~\cite[7]{Abr02}). Eine Form dieser zweiwertigen Logik wird später von dem französischen Psychoanalytiker Jacques Lacan (1901 - 1981) in seine Theorien einbezogen~\cite[317]{Liu10}
    }
\end{enumerate}


Die Ähnlichkeiten zu dem \textbf{Alles-oder-Nichts-Prinzip} von Nervenzellen zur Formalisierung seines Kalküls berücksichtigt er ab 1929:

\blockquote[{~\cite[6]{Mcc61}}]{
    [...] it dawned on me that these events might be regarded as the all-or-none-impulses of neurons [...]. I began to try to formulate a proper calculus for these events by subscripting symbols for propositions in some sort of calculus of propositions [...]
}

Knapp 14 Jahre später wird McCulloch in seinem Papier das ``psychon`` mit der Aktivität eines einzelnen Neurons gleichsetzen:

\blockquote[{~\cite[114]{MP43}}]{
    [...] a psychon can be no less than the activity of a single neuron.
}

Doch vorerst stagniert seine Arbeit, und erst, als er nach seiner Berufung an das Neuropsychiatrische Institut der Universität von Illinois (Chicago) im Jahr 1941 Walter Pitts kennenlernt, ist er in der Lage, den Kalkül zu vervollständigen.\\

Pitts besucht dort Kurse bei dem Philosophen und Logiker Rudolf Carnap (1891 - 1970), außerdem bei dem Biomathematiker\footnote{
    ``Biomathematik`` ist ein Teilgebiet der angewandten Mathematik und die mathematische Ausrichtung der theoretischen Biologie, die sich mit Modell- und Theoriebildung zur Beschreibung biologischer Zusammenhänge beschäftigt. ({\url{https://de.wikipedia.org/wiki/Theoretische\_Biologie}}, abgerufen 09.08.2023). Teilgebiete der Biomathematik sind u.a. Demographie, Mathematische Ökologie, Epidemologie, Populationsgenetik
} Nicolas Rashevsky (1899 - 1972) [Pic04:183 letzter Abs. ff.], der in seiner Arbeit mathematische Beschreibungen für die Modellierung von Funktionen von Nerven und Nervennetzen verwendet~\cite[13]{Abr02}\footnote{
    ``In these works, formal logic and mathematics played a strong role, and thus biology as a discipline was reaching a more mature stage, as it began to incorporate the scientific method of physics, that is, using theoretical analysis and mathematical formulations.``~\cite[7]{Abr02}
} sowie ab 1939 die Zeitschrift ``Bulletin of Mathematical Biophysics`` veröffentlicht~\cite[16]{Abr02} (in dieser wird später~\cite{MP43} erstmalig publiziert).
Der knapp 19-jährige Pitts gilt als brillianter Mathematiker\footnote{
    Siehe ~\cite[4]{Arb19}, außerdem~\cite[22]{Abr02}: ``Pitts, a mathematical prodigy``
} und später als ``the intelligence behind [...] McCulloch``: Gemeinsam mit ihm veröffentlicht McCulloch nur wenigen Monaten nach ihrem Kennenlernen das ``logical calculus``-Papier~\cite[104]{AR98}


\section{Norbert Wiener}\label{appendix:wiener}

Norber Wiener (1894 - 1964) gilt als Begründer der Kybernetik, die \textit{Küppers} beschreibt als

\blockquote[{~\cite[2]{Kup19}}]{
    wirkungsvolle Kommunikation bzw. verlustarmer Daten- und Informationsaustausch, der in der Natur die Überlebensfähigkeit und in der Technik die maschinelle, prozessuale Funktionalität stärkt und dadurch Fehler vermeiden hilft.
}

Wiener und McCulloch (s. \ref{appendix:mcculloch}) lernen sich im Winter 1942 kennen~\cite[218]{Mas90}.
Unter dem Vorsitz von McCulloch wird auf den sog. ``Macy-Konferenzen``\footnote{
    \url{https://www.asc-cybernetics.org/foundations/history/MacySummary.htm}, abgerufen 09.08.2023
} zwischen 1946 und 1953 Biologie und Technologie diskutiert~\cite[5]{Arb19}.
In Folge dessen veröffentlicht Wiener 1948 seine Arbeit ``Cybernetics: Or Control and Communication in the Animal and the Machine`` [Wie48].